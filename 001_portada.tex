%
% Portada.
%

% Nota: Sería más cómodo emplear el comando \maketitle que genera una portada de forma automática, pero
% no incluye toda la información que es necesario incluir en la memoria de un proyecto de fin de carrera
% de la Facultad de Informática de A Coruña.
%

\begin{titlepage}

	\begin{center}

		% Logotipo de la universidad.
		\includegraphics[width=6cm]{./eps/logo_udc-eps-converted-to.pdf}
		\vspace{2cm}

		{\Large{\textbf{Facultade de Informática da Universidade de A Coruña}}}
		\\
		{\it \large{\textbf{Computación}}}
		\vspace{1cm}

		% Indicamos el nombre de la titulación oficial que hemos cursado con tanto esfuerzo.
		{\large PROYECTO DE FIN DE CARRERA\\Ingeniería Informática}
		\vspace{1cm}

		% Título
		\textbf{\Large Desarrollo de un videojuego roguelike para invidentes aplicando técnicas de Procesamiento del Lenguaje Natural.}
		\vspace{7cm}
	\end{center}

	\begin{flushright}
		\begin{tabular}{ll}
			\large{\textbf{Alumno:}}	&
			\large{Darío Penas Sabín} \\

			% Nombre del director/tutor del proyecto.
			\large{\textbf{Director:}}	&
			\large{Jesús Vilares Ferro} \\

			% Nombre del director/tutor del proyecto.
			\large{\textbf{Director:}}	&
			\large{Carlos Gómez Rodríguez} \\

			% Fecha.
			\large{\textbf{Fecha:}}	&
			\large{CHANGE: 15 de junio de 2016} \\
		\end{tabular}
	\end{flushright}

\end{titlepage}
