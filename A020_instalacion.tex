\chapter{Instalación e Instrucciones}
\label{ref:instalacion}

El \textit{roguelike} es software libre y el código fuente se puede encontrar en Github\footnote{\url{https://github.com/dpenas/roomsgame}}, al igual que varias versiones ejecutables. Dichas versiones ejecutables contienen un archivo .jar que solamente requiere tener una versión de JRE\footnote{\url{http://www.oracle.com/technetwork/java/javase/downloads/jre8-downloads-2133155.html}}{\footnote{Java Runtime Environment} superior a la 6 instalada en el sistema a usar. 

Para su ejecución basta con hacer doble clic en el archivo .jar o ir al directorio donde se encuentre dicho archivo e introducir:

\begin{lstlisting}[label=lst:bash,caption=Comando para la ejecución del videojuego]
java -jar game.jar
\end{lstlisting}

Para cambiar el idioma del proyecto se debe de localizar el archivo languages.properties y cambiar el idioma a ES, GL o EN para obtenerlo en español, gallego o inglés, respectivamente. También es posible cambiar las teclas del propio juego. Las que vienen por defecto se puede encontrar en la wiki del proyecto.\footnote{\url{https://github.com/dpenas/roomsgame/wiki}}.

En algunas ocasiones las frases generadas no se leen directamente. Esto viene causado porque no tener activado el \textit{Java Access Bridge} o no está instalado la versión correcta (se recomienda tener instalada la versión de 32 y 64 bits para evitar problemas). 
En sistemas Windows \textit{Java Access Bridge} se puede activar fácilmente de la forma siguiente:

\begin{itemize}
\item Ir a \textbf{Start > Control Panel > Ease of Access > Ease of Access Center}
\item Seleccionar \textbf{Use the computer without a display}
\item En la sección \textbf{Other programs installed}, seleccionar: \textbf{Enable Java Access Bridge}
\end{itemize}

Estas instrucciones deberían de servir para Windows Vista y posteriores. En caso contrario, también se podrá hacer:

\begin{verbatim}
%JRE_HOME%\bin\jabswitch -enable
\end{verbatim}

donde \textit{\%JRE\_HOME\%} es el directorio donde Java está instalado.

Para otros sistemas operativos será necesario la descarga y activación de manera manual.  \footnote{\url{http://www.oracle.com/technetwork/articles/javase/index-jsp-136191.html}}}