\chapter{Instalación e Instrucciones del Juego}
\label{ref:instalacion}

Nuestro \textit{roguelike} es software libre y el código fuente está libremente disponible en Github,\footnote{\url{https://github.com/dpenas/roomsgame}} al igual que varias versiones ejecutables del mismo. Dichas versiones ejecutables contienen un archivo \texttt{.jar} que solamente requiere tener una versión de JRE\footnote{\url{http://www.oracle.com/technetwork/java/javase/downloads/jre8-downloads-2133155.html}}{\footnote{Java Runtime Environment} superior a la 6 instalada en el sistema a usar. 

Para su ejecución basta con hacer doble clic en el archivo \texttt{.jar} o ir al directorio donde se encuentre dicho archivo e introducir:

\begin{lstlisting}[label=lst:bash,caption=Comando para la ejecución del videojuego]
java -jar game.jar
\end{lstlisting}

Para cambiar el idioma del juego accederemos al archivo \texttt{languages.properties} y cambiaremos el idioma a ES, GL, EN o NL para jugar en español, gallego, inglés u holandés, respectivamente. También es posible cambiar las teclas del propio juego. Las que vienen por defecto se puede encontrar en la wiki del proyecto.\footnote{\url{https://github.com/dpenas/roomsgame/wiki}}

En algunas ocasiones las frases generadas no se leen directamente. Esto viene causado por no tener activado el \textit{Java Access Bridge} o si no está instalada la versión correcta del mismo (se recomienda tener instalada la versión de 32 y 64 bits para evitar problemas). 
En sistemas Windows \textit{Java Access Bridge} se puede activar fácilmente de la forma siguiente:

\begin{itemize}
\item Ir a \textbf{Start \textrightarrow Control Panel \textrightarrow Ease of Access \textrightarrow Ease of Access Center}
\item Seleccionar \textbf{Use the computer without a display}
\item En la sección \textbf{Other programs installed}, seleccionar: \textbf{Enable Java Access Bridge}
\end{itemize}

\noindent Estas instrucciones deberían servir para Windows Vista y posteriores. En caso contrario, también se podrá hacer:

\begin{verbatim}
%JRE_HOME%\bin\jabswitch -enable
\end{verbatim}

donde \textit{\%JRE\_HOME\%} es el directorio donde Java está instalado.

\noindent Para otros sistemas operativos será necesario la descarga y activación de manera manual.\footnote{\url{http://www.oracle.com/technetwork/articles/javase/index-jsp-136191.html}}}