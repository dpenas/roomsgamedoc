\documentclass[a4paper,11pt,twoside,openright]{book}

\usepackage[spanish]{babel}
\usepackage[utf8]{inputenc}

\usepackage{eurosym}
\usepackage{fancybox}
\usepackage{multicol}
\usepackage{amsmath}
\usepackage{subcaption}
\usepackage[nottoc,numbib]{tocbibind}
\usepackage[pdftex,usenames,dvipsnames]{color}
\usepackage{float}
% compilar con pdflatex.
\usepackage[pdftex]{graphicx}


\usepackage{setspace}
\onehalfspacing
\usepackage{geometry}
\geometry{
  inner=3.5cm,
  outer=2.5cm,
  bottom=3.5cm,
  top=3.5cm}

\usepackage{fancyhdr}
\pagestyle{fancy}
\fancyhf{}
\fancyhf[HR]{\thepage}
\fancyhf[HL]{\nouppercase\rightmark}

\usepackage{booktabs}
\usepackage{hyperref}
\hypersetup{
    colorlinks,
    linkcolor={red!50!black},
    citecolor={blue!50!black},
    urlcolor={blue!80!black}
}
\usepackage{rotating}
\usepackage{multicol}
\usepackage{multirow}
\usepackage{pgfgantt}
\usepackage{enumitem}
\usepackage{color}
\usepackage{xcolor}
\usepackage{caption}
\DeclareCaptionFont{white}{\color{white}}
\DeclareCaptionFormat{listing}{\colorbox{gray}{\parbox{\textwidth}{#1#2#3}}}
\captionsetup[lstlisting]{format=listing,labelfont=white,textfont=white}

 \usepackage{listings}
 \usepackage{courier}
 \lstset{
         basicstyle=\footnotesize\ttfamily, % Standardschrift
         numberstyle=\tiny,          % Stil der Zeilennummern
         numbersep=5pt,              % Abstand der Nummern zum Text
         tabsize=2,                  % Groesse von Tabs
         extendedchars=true,         %
         breaklines=true,            % Zeilen werden Umgebrochen
         keywordstyle=\textbf,
         frame=b,
         stringstyle=\textit, % Farbe der String
         showspaces=false,           % Leerzeichen anzeigen ?
         showtabs=false,             % Tabs anzeigen ?
         xleftmargin=17pt,
         framexleftmargin=17pt,
         framexrightmargin=5pt,
         framexbottommargin=4pt,
         %backgroundcolor=\color{lightgray},
         showstringspaces=false      % Leerzeichen in Strings anzeigen ?
 }

 \lstset{literate=
  {á}{{\'a}}1 {é}{{\'e}}1 {í}{{\'i}}1 {ó}{{\'o}}1 {ú}{{\'u}}1
  {Á}{{\'A}}1 {É}{{\'E}}1 {Í}{{\'I}}1 {Ó}{{\'O}}1 {Ú}{{\'U}}1
  {à}{{\`a}}1 {è}{{\`e}}1 {ì}{{\`i}}1 {ò}{{\`o}}1 {ù}{{\`u}}1
  {À}{{\`A}}1 {È}{{\'E}}1 {Ì}{{\`I}}1 {Ò}{{\`O}}1 {Ù}{{\`U}}1
  {ä}{{\"a}}1 {ë}{{\"e}}1 {ï}{{\"i}}1 {ö}{{\"o}}1 {ü}{{\"u}}1
  {Ä}{{\"A}}1 {Ë}{{\"E}}1 {Ï}{{\"I}}1 {Ö}{{\"O}}1 {Ü}{{\"U}}1
  {â}{{\^a}}1 {ê}{{\^e}}1 {î}{{\^i}}1 {ô}{{\^o}}1 {û}{{\^u}}1
  {Â}{{\^A}}1 {Ê}{{\^E}}1 {Î}{{\^I}}1 {Ô}{{\^O}}1 {Û}{{\^U}}1
  {œ}{{\oe}}1 {Œ}{{\OE}}1 {æ}{{\ae}}1 {Æ}{{\AE}}1 {ß}{{\ss}}1
  {ç}{{\c c}}1 {Ç}{{\c C}}1 {ø}{{\o}}1 {å}{{\r a}}1 {Å}{{\r A}}1
  {€}{{\EUR}}1 {£}{{\pounds}}1
 }

\parskip=6pt

\parindent=10pt

\setcounter{secnumdepth}{3}

\title{Desarrollo de un videojuego roguelike para invidentes aplicando técnicas de Procesamiento del Lenguaje Natural.}
\author{Darío Penas Sabín}
\date{\today}


\renewcommand\lstlistingname{Fragmento de Código}
\renewcommand\lstlistlistingname{Fragmentos de Código}

% START JSON coloring

\newcommand\JSONnumbervaluestyle{\color{blue}}
\newcommand\JSONstringvaluestyle{\color{red}}

% switch used as state variable
\newif\ifcolonfoundonthisline

\makeatletter

\lstdefinestyle{json}
{
  showstringspaces    = false,
  keywords            = {false,true},
  alsoletter          = 0123456789.,
  morestring          = [s]{"}{"},
  stringstyle         = \ifcolonfoundonthisline\JSONstringvaluestyle\fi,
  MoreSelectCharTable =%
    \lst@DefSaveDef{`:}\colon@json{\processColon@json},
  basicstyle          = \ttfamily,
  keywordstyle        = \ttfamily\bfseries,
}

% flip the switch if a colon is found in Pmode
\newcommand\processColon@json{%
  \colon@json%
  \ifnum\lst@mode=\lst@Pmode%
    \global\colonfoundonthislinetrue%
  \fi
}

\lst@AddToHook{Output}{%
  \ifcolonfoundonthisline%
    \ifnum\lst@mode=\lst@Pmode%
      \def\lst@thestyle{\JSONnumbervaluestyle}%
    \fi
  \fi
  %override by keyword style if a keyword is detected!
  \lsthk@DetectKeywords% 
}

% reset the switch at the end of line
\lst@AddToHook{EOL}%
  {\global\colonfoundonthislinefalse}

\makeatother

% END JSON coloring

\newenvironment{bottompar}{\par\vspace*{\fill}}{\clearpage}

\begin{document}
  \lstset{basicstyle=\ttfamily}
        %
% Portada.
%

% Nota: Sería más cómodo emplear el comando \maketitle que genera una portada de forma automática, pero
% no incluye toda la información que es necesario incluir en la memoria de un proyecto de fin de carrera
% de la Facultad de Informática de A Coruña.
%

\begin{titlepage}

	\begin{center}

		% Logotipo de la universidad.
		\includegraphics[width=6cm]{./eps/logo_udc-eps-converted-to.pdf}
		\vspace{2cm}

		{\Large{\textbf{Facultade de Informática da Universidade de A Coruña}}}
		\\
		{\it \large{\textbf{Computación}}}
		\vspace{1cm}

		% Indicamos el nombre de la titulación oficial que hemos cursado con tanto esfuerzo.
		{\large PROYECTO DE FIN DE CARRERA\\Ingeniería Informática}
		\vspace{1cm}

		% Título
		\textbf{\Large Desarrollo de un videojuego roguelike para invidentes aplicando técnicas de Procesamiento del Lenguaje Natural.}
		\vspace{7cm}
	\end{center}

	\begin{flushright}
		\begin{tabular}{ll}
			\large{\textbf{Alumno:}}	&
			\large{Darío Penas Sabín} \\

			% Nombre del director/tutor del proyecto.
			\large{\textbf{Director:}}	&
			\large{Jesús Vilares Ferro} \\

			% Nombre del director/tutor del proyecto.
			\large{\textbf{Director:}}	&
			\large{Carlos Gómez Rodríguez} \\

			% Fecha.
			\large{\textbf{Fecha:}}	&
			\large{CHANGE: 15 de junio de 2016} \\
		\end{tabular}
	\end{flushright}

\end{titlepage}


        \frontmatter

        \thispagestyle{empty}

        \section*{Resumen:}

La industria del entretenimiento digital ha crecido inmensamente en los últimos años, llegando a alcanzar números de ventas jamás vistos anteriormente.
Parte de la razón de este crecimiento viene dada por una mejora radical en el aspecto visual, necesaria para que el jugador se sienta inmerso en la aventura que se le está planteando.
Estas mejoras, sin embargo, dejan de lado a muchos jugadores que, por diferentes motivos, no son capaces de apreciar el contenido visual que se les ofrece o tienen problemas para ello, haciendo imposible que disfruten del contenido ofrecido.

Este proyecto consiste en la creación de \textit{The accessible dungeon}, un videojuego de género \textit{roguelike} para invidentes que, desde un principio, parte de la idea de generar contenido específicamente diseñado para que pueda ser jugado por todo el mundo, haciendo énfasis en ofrecer al jugador una descripción textual, expresiva y variada de lo que está sucediendo en su alrededor. Dicha descripción será generada automáticamente en base a una serie de gramáticas y diccionarios, empleando para ello técnicas de Procesamiento del Lenguaje Natural.

        \thispagestyle{empty}

        \section*{Lista de palabras clave:}

\begin{itemize}
  \item Tiflotecnología
  \item Procesamiento del Lenguaje Natural
  \item Accesibilidad
  \item Entretenimiento Digital
  \item \textit{Roguelike}
  \item Java
  \item \textit{Open Source}
\end{itemize}

        \thispagestyle{empty}

        %
% Agradecimientos
%

\section*{Agradecimientos}

TODO


\begin{flushright}
  Darío Penas Sabín \\
  Amsterdam, PONER FECHA
\end{flushright}


        \thispagestyle{empty}

        \tableofcontents
        \listoffigures

        \mainmatter
        \chapter[Introdución]{Introducción}

En este capítulo introductorio se explicarán los aspectos necesarios para entender lo más importante del proyecto, la motivación para la realización del mismo y un breve resumen del resto de capítulos que forman parte de la memoria.
Cabe destacar que todas las imágenes mostradas en este proyecto son de dominio público.

\section{Videojuegos y personas invidentes}
La mayor parte de los videojuegos comerciales no tienen en cuenta a muchas minorías de la sociedad. Haciendo una pequeña búsqueda online pueden encontrarse miles de usuarios quejándose de \textit{first person shooters. Género de videojuegos en primera persona donde el usuario dispara con diferentes armas a enemigos.} que tienen un \textit{campo de visión} limitado, que termina causándoles mareos al poco rato; usuarios zurdos que tienen que acomodarse a ciertos controles a no existir una opción para cambiarlos; usuarios daltónicos protestando que diferentes juegos (como por ejemplo The Witness\footnote{Juego de puzzles en primera persona: \url{http://the-witness.net/}}), basan buena parte de su mecánica en que el jugador sea capaz de distinguir diferentes colores; usuarios invidentes que no pueden disfrutar de prácticamente ninguno de los títulos que se encuentran en el mercado, etc.

En este proyecto nuestro objetivo ha sido el crear un videojuego desde cero que tenga en cuenta la problemática de las personas con deficiencias visuales, tomando especial relevancia el desarrollo para invidentes y centrándose en los aspectos que sean relevantes para ellos.

\subsection{Visita a la \textit{ONCE}}
El pasado curso, junto a uno de mis co-directores, asistí de tiflotecnología organizado por la Organización Nacional de Ciegos Españoles (ONCE) donde, entre otros temas, uno de sus formadores nos habló sobre la tiflotecnología, nos mostró la forma en la que usaban ordenadores y teléfonos móviles, incluso para leer código fuente y los errores, muchos de ellos fácilmente solventables, que se cometían día a día en temas de accesibilidad por parte de los desarrolladores. Esta visita nos resultó muy interesante y gracias a ella fuimos capaces de detectar posibles mejoras que tener en cuenta en el proyecto y los fallos que no deberíamos cometer en su diseño. También se ofrecieron a probar el proyecto una vez estuviera listo, pero este tema será tratado en las próximas secciones.

\section{Motivación}
A nivel personal, como la mayor parte de las personas de mi generación actual, el entretenimiento digital siempre ha sido parte de mi vida y, en mi caso, una de las razones por las que desde pequeño estuve interesado en la informática y motivo por el que, finalmente, acabé estudiando esta carrera. Del mismo modo, y habiéndome relacionado con bastante gente con toda clase de necesidades especiales durante un gran periodo de tiempo, mi interés por la tiflotecnología y las limitaciones que la tecnología ofrece a millones de personas no hizo más que crecer año a año.

\section{Estructura de la memoria}

La memoria está formada por ocho capítulos en los que se explican los pasos tomados a la hora de crear el proyecto.

\paragraph*{Capítulo 1. Introducción.}
El presente capítulo, donde se explican, de manera general y resumida, en qué consiste el proyecto, la motivación del mismo y la estructura de la memoria.

\paragraph*{Capítulo 2. Estado del Arte.}
En este apartado hablaremos de otros proyectos similares, de la situación actual de la industria sobre el problema que tratamos en este proyecto y del diferente software que es utilizado en relación con la tiflotecnología.

\paragraph*{Capítulo 3. Fundamentos Tecnológicos.}
Trataremos las herramientas y recursos empleados durante la elaboración del proyecto.

\paragraph*{Capítulo 4. Metodología.}
Se detallarán las prácticas y metodologías de desarrollo empleadas y la razón de su uso.

\paragraph*{Capítulo 5. Planificación y Seguimento.}
Detallaremos la planificación y seguimiento de cada una de las etapas del proyecto. 

\paragraph*{Capítulo 6. Análisis de Requisitos.}
Se comentará el análisis de requisitos para este proyecto y se explicará en detalle cada uno de los mismos.

\paragraph*{Capítulo 7. Diseño e Implementación.}
En este capítulo explicaremos los detalles del diseño e implementación del sistema, centrándonos especialmente en aquellas partes del programa que consideramos más relevantes.

\paragraph*{Capítulo 8. Conclusión y trabajo futuro.}
En este apartado se comentarán las conclusiones obtenidas y se detallarán los posibles cambios, mejoras y extras que se
podrían tener en cuenta en el futuro para su mejora.
        \chapter{Estado del Arte}

En este capítulo hablaremos, principalmente, de dos importantes aspectos. El primero es la situación de la industria del entretenimiento digital hoy en día, seguido de una definición sobre lo que es un \textit{roguelike}. El segundo son los elementos que dificultan y facilitan el uso de diferentes programas y \textit{software} a ciertos sectores de la sociedad como invidentes o daltónicos; cómo algunos programas intentan solventar estos problemas y las razones por las que hemos elegido introducir ciertos elementos en nuestro proyecto para diliar con los mismos.

\section{La industria del entretenimiento digital en la actualidad}

Desde sus primeros pasos hasta hoy en día, tal y como sucede con muchas de las novedades en el mundo del entretenimiento y la cultura, el sector del ocio digital ha sufrido cierto estigma por una gran parte de la población, siendo censurado y degragado en mayor o menor medida, no tan solo por cierta parte de la sociedad al considerarlo como algo enfocado para niños y sin mucho interés o relevancia cultural, pero también por muchos medios de comunicación y gobiernos. A pesar de que hoy en día este problema todavía está activo\footnote{Es común que cada año en Australia se censuren algunos juegos como \href{http://goo.gl/hFrQah}{Paranautical Activity} por razones que otras formas de entretenimiento y cultura como películas o libros no se ven tan afectados.}, la industria se ha expandido tanto (consolas, ordenadores, navegadores, Facebook, móvil...), que cada vez es más complicado encontrar a alguien que no haya jugado a algún videojuego en las últimas semanas, ya es un elemento que forma parte del día a día de mucha parte de la población y defenderlo como un elemento de gran importancia cultural es cada vez más sencillo.

\subsection{La industria, en números}
En todo el mundo, pero especialmente en EEUU, la industria de los videojuegos es uno de los sectores con más crecimiento \cite{website:gamesimprovingeconomy} llegando a generar, solamente en ventas digitales, alrededor de 61 billones de dólares en el año 2015 \cite{website:gamingsales}.

Este gran éxito se debe, en gran parte, a la irrupción de los juegos desarrollados para móviles, cuyo beneficio ha ido aumentando enormemente durante los últimos años. \footnote{\href{http://goo.gl/Lz9UAa}{La venta de videojuegos en Alemania crece año tras año, pero el mayor aumento de beneficio se está centrando en el mercado de los juegos para móvil}}. Sin embargo, esto no significa que el resto de plataformas no estén triunfando o se hayan quedado estancadas. Solamente Steam, la plataforma de distribución digital para PC por excelencia que ha sido desarrollada por Valve, ha generado alrededor de 3 billones y medio de dólares en el año 2015\cite{website:steamgamesmarket}.

También hemos visto nuevas consolas salir al mercado hace poco más de dos años: PlayStation 4 y Xbox One. La primera de ellas ha logrado vender, a principios del año 2016, alrededor de 36 millones de unidades\cite{website:ps4sales}, mientras que la consola de Microsoft llega a los 19.1 millones en el mismo periodo\cite{xboxonesales}, haciendo un total de más de 55 millones de unidades en total, la cual es una gran cifra en comparación con los años anteriores.

Con el mercado del PC resurgiendo, las consolas de sobremesa obteniendo grandes números de ventas, las portátiles resistiendo la lucha contra los móviles, el mercado de los videojuegos para móvil en esplendor y los cascos de realidad virtual llegando al mercado este año 2016 y obteniendo grandes números de preventas; todo parece indicar que la industria de ocio digital no hará más que crecer durante los próximos años.

\subsection{Conclusión}

Lo que comenzó hace varias décadas como un modo de entretenimiento sin ninguna pretensión, generalmente enfocado a un público infantil o adolescente y que miraba a otras industrias como la cinematográfica con recelo, tanto en números como en visibilidad, se ha convertido en todo lo que había deseado y más. Gracias a grandes títulos y a su expansión a toda clase de dispositivos, no se puede hablar de la industria del entretenimiento sin hablar de videojuegos y, en muchos casos, algunos de esos títulos han logrado ser nombrados como obras de arte en su género, pasando a la historia y siendo recordados a lo largo de los años.

\section{\textit{Roguelikes.}}
\label{sec:roguelikeinformacion}

\subsection{Qué es y orígenes}

En 1983, Michael Toy y Glenn Wichman crearon un videojuego llamado Rogue\footnote{Desde 2014 este juego se encuentra disponible en \href{https://archive.org/details/msdos_Rogue_1983}{archive.org}} que acabó definiendo un género que sigue vigente y en gran esplendor hoy en día.

Las características principales que definieron a Rogue y que, por extensión, definieron al género de los \textit{roguelikes} inicialmente son:

\paragraph{Dificultad}: Rogue es un videojuego difícil con \textit{permadeath}\footnote{Una vez que el jugador muere, tiene que empezar desde el principio; no hay partidas guardadas} que obligará al jugador a rejugarlo una y otra vez, intentando llegar más lejos que la anterior partida gracias a ir aprendiendo los funcionamientos del mismo.

\paragraph{Aleatoriedad}: Cada vez que el jugador comienza una partida nueva se encontrará con ciertos elementos que han cambiado con respecto a la vez anterior: el mapa es distinto, los elementos y enemigos se encuentran en sitios diferentes, los objetos obtenidos han cambiado... causando que cada vez que el usuario empiece, tenga un grado de dificultad pseudo-aleatorio dependiendo de la semilla con la que estos elementos hayan sido generados.

\paragraph{Progresión}: Una de las frases más escuchadas en las críticas que Rogue recibió tras su lanzamiento es que el jugador sentía la necesidad de intentar llegar más lejos en cada ocasión\cite{website:machinesnetworks}. Esto viene dado, sobre todo, por la sensación de progresión y de que en cada \textit{run}\footnote{Palabra comúnmente usada en estos géneros y que se refiere a una partida desde su inicio hasta que el jugador pierde} el usuario vaya mejorando. Dentro de la propia partida también existe una progresión a medida que el usuario derrota enemigos, consiguiendo puntos de experiencia, subiendo niveles y obteniendo mejores armas y armaduras con las que ser un poco más fuerte. La intriga por saber los nuevos objetos que se pueden conseguir, así como los nuevos enemigos con los que nos enfrentaremos y mapas que se generarán, hace que \textit{Rogue} fuera fácilmente rejugable e interesante para los usuarios.

\begin{figure}[h!]
		\includegraphics[width=\textwidth,height=\textheight,keepaspectratio]{./img/roguegame.PNG}
	\caption{Captura de pantalla de \href{https://en.wikipedia.org/wiki/File:Rogue_Unix_Screenshot_CAR.PNG}{dominio público} (tal y como todas las imágenes mostradas en este proyecto) del videojuego Rogue}
	\label{fig:roguegame}
\end{figure}

A partir de este momento muchos fueron los juegos que decidieron imitar estas características de Rogue, añadiendo, cambiando o enfatizando diferentes elementos, por eso se han denominado \textit{roguelikes}.

\subsection{En la actualidad}

Tras el éxito de Rogue, fueron muchos los títulos que simularon su fórmula de éxito e intentaron mejorarlo, sobre todo gráficamente. Algunos de ellos se centran en diferentes aspectos (combate en vez de exploración, por ejemplo) y llegan a ser completamente diferentes a la hora de jugarlos (por turnos o tiempo real) pero, sin embargo, todos conservan buena parte de las características que hicieron al género famoso hasta hoy en día.

\begin{figure}[h!]
		\includegraphics[width=\textwidth,height=\textheight,keepaspectratio]{./img/Vultures.jpg}
	\caption{Captura de pantalla del videojuego Vultures}
	\label{fig:vulturesgame}
\end{figure}

Incluso muchos de los nuevos títulos, muchos de ellos desarrollados por empresas independientes, no dan especial importancia a tener gráficos en 3D o texturas increíbles, sino que optan por profundizar en crear una ambientación interesante y ofrecer un sistema de combate y exploración únicos y pulidos.
Uno de ellos que ha triunfado enormemente fue FTL. Hemos adjuntado una captura de pantalla \ref{fig:ftl} como referencia.

\begin{figure}[h!]
		\includegraphics[width=\textwidth,height=\textheight,keepaspectratio]{./img/ftl.jpg}
	\caption{Captura de pantalla del videojuego FTL: Faster Than Light}
	\label{fig:ftl}
\end{figure}

\subsubsection{La creación de subgéneros}

Dado que perder todo el progreso y tener que empezar desde el principio sin haber conseguido nada más que la experiencia personal es algo que no atrae a mucha gente hoy en día, son numerosos los juegos que han añadido más elementos de progreso general para que el jugador no se sienta frustrado. Estos elementos pueden ser nuevos personajes con los que jugar, puntos de experiencia o dinero con lo que poder equipar y mejorar desde un principio a nuestro personaje para poder llegar más lejos que la anterior vez, diferentes modos de juegos que se desbloquean al llegar a una cierta puntuación, etc.
También es común ver juegos que se basan en partidas cortas, de como mucho una hora, para que la repetición sea mayor y perder el personaje no sea un gran ``castigo'' que tener que afrontar.

Estos cambios que se han realizado durante los últimos años y que, de cierta manera, han modificado el género que Rogue creó en un inicio, no siempre se han tomado positivamente por parte de la comunidad, que se suele quejar de que muchos títulos que se definen a sí mismos como \textit{roguelike} no contienen ciertas características, como la dificultad, que una vez definieron el género. Por este motivo se han definido subgéneros como el \textit{roguelite}, que toman muchas de esas ideas iniciales, pero añaden o ignoran otras muchas para crear un título que sea un poco más sencillo y no penalice tanto al jugador.

\subsection{Elementos \textit{roguelike} en nuestro proyecto}

En nuestro caso hemos creado un \textit{roguelike} similar a Rogue, no solamente estéticamente, pero también en diseño y funcionalidad. El usuario se moverá por un mapa aleatoriamente generado y luchará contra diferentes enemigos que intentarán eliminarlo de diferentes formas. En base al nivel que el usuario tenga los enemigos serán más o menos complicados de batir y la recompensa por hacerlo será mayor.

El objectivo del juego es llegar lo más lejos posible dentro de la mazmorra. Cada vez que el jugador entra en un portal se añadirá un punto (el número de puntos se mostrará en la pantalla) y, cada vez que esto suceda, un nuevo mapa, con diferentes características y contenido, será generado. El juego es complicado, aleatorio y con una sensación de progreso, tal y como el género \textit{roguelike} especifica.

\begin{figure}[H]
		\includegraphics[width=140mm, height=185mm]{./img/roomsGameGeneral.png}
	\caption{Captura de pantalla de la interfaz de usuario de nuestro juego}
	\label{fig:roomsgamegeneral}
\end{figure}

\section{Problemas de la tecnología para ciertos sectores de la sociedad}
\label{sec:dificultadesfeedback}

Algunas de las razones que hacen de la tecnología un elemento prácticamente necesario hoy en día y que facilita su usabilidad crea, paradójicamente, una barrera para mucha gente que no puede disfrutar de ella.

En esta sección hablaremos sobre algunos de los problemas que tantos invidentes como daltónicos se encuentran actualmente en diferentes programas y cómo, en bastantes ocasiones, solventarlos no es muy complicado, lo que clarifica que muchas de estas limitaciones son dadas más por el desconocimiento que por la dificultad de la implentación de una solución.

\subsection{Dificultades a los que se enfrentan los invidentes}

Las dificultades que tienen invidentes o personas con diferentes grados de ceguera en nuestra sociedad es enorme y, tristemente, los dispositivos tecnológicos y software no se ven excluidos de esta lista. 
No poder hacer uso de una interfaz gráfica o ver lo que está sucediendo en la pantalla es un problema que automáticamente imposibiliza el uso de la mayoría de las aplicaciones y videojuegos en su totalidad. Incluso navegar por internet, donde la mayoría del contenido es texto que puede ser leído fácilmente por un lector de pantalla, puede llegar a ser un quebradero de cabeza si cierta parte de ese contenido está en flash (el lector no podrá hacer nada en este caso), los enlaces a redes sociales están en posiciones poco recomendables (dado que su contenido se puede llegar a reproducir y generalmente son un suplicio saltarlos), algunos de los nombres usados en programas software son poco descriptivos y dificultan su correcta identificación, etc.

\subsection{Cómo solventar parte de estos problemas para los invidentes}

Uno de las primeros problemas que nos comentaron en la charla de la ONCE es que muchos de los desarrolladores que programan algo específicamente para ciegos no tienen en cuenta que, la mayoría de las veces, hay gente que puede ver a su alrededor que les puede describir lo que está sucediendo en la pantalla, por lo que es muy importante crear una interfaz que esté actualizada y muestre la misma información que la persona invidente reciba.
También es necesario dar la posibilidad al usuario de repetir el contenido generado, dado que algunos lectores tienen problemas para leer ciertos caracteres o lo que leen no es del todo claro y, por lo tanto, debe de ser fácilmente repetido.

Algunos videojuegos han visto versiones adaptadas para invidentes, la mayoría de ellas desarrolladas por la comunidad, o nuevos títulos que se centran en ofrecer una experiencia revolucionaria desde el principio. Un buen ejemplo de ello es Shades of Doom\footnote{\url{http://www.gmagames.com/sod.html}}, que solamente usa sonido (principalmente ruidos repetitivos) y muy pocas frases para que el jugador sea capaz de descubrir lo que tiene que hacer y dónde está.

\subsection{Cómo hemos solventado el problema en nuestro proyecto}

En el proyecto, dado que se basa en la generación pseudo-aleatoria de frases, hemos creado descripciones para todos los elementos de la pantalla, por lo que el jugador siempre puede saber dónde se encuentra, qué hay a su alrededor, cuáles son sus características, etc. Hemos creado una ventana que se encuentra al lado del juego y donde se van guardando todas las frases generadas, siendo la primera frase la última generada y leyéndose automáticamente por el reproductor de pantalla que esté siendo usado en ese momento. De esta forma, una persona con visión siempre puede leer dicha frase y el jugador repetirla las veces que quiera.

\begin{figure}[H]
		\includegraphics[width=\textwidth,height=\textheight,keepaspectratio]{./img/roomsGameTextArea.png}
	\caption{Captura de pantalla del area donde mostramos las frases generadas de nuestro juego para los invidentes}
	\label{fig:roomsgametextarea}
\end{figure}

\subsection{Dificultades a los que se enfrentan los daltónicos}

Daltonismo es un defecto genérico que afecta, aproximadamente, al 1\% de la población. Lo que en un principio parece como un pequeño inconveniente que no cambia mucho la vida de la persona que lo sufre, lo cierto es que hay muchas ocasiones en las que incluso ver un simple partido de fútbol americano correctamente puede convertirse en algo casi imposible para ellos \footnote{En \href{http://goo.gl/o3GkrP}{algunos partidos} los jugadores llevan camisetas con colores problemáticos.}. Incluso ver un mero mapa puede ser una complicación\footnote{\href{https://i.imgur.com/CMCywUU.jpg}{Resumen de algunos problemas que tienen los daltónicos a ver mapas}}.

En el mercado del ocio digital, donde los gráficos y paletas de colores juegan un gran papel en el arte del título, este problema se ve acentuado. La peor parte viene si parte de las mecánicas del juego necesitan que el jugador sea capaz de distinguir colores para obtener cierta información relevante; y esto es algo que sucede en numerosos títulos. En algunos casos es una pequeña molestia como en \textit{Borderlands}\footnote{Juego de acción en primera persona lanzado inicialmente en 2009 y que cuenta con varias secuelas} donde hay armas especiales que se diferencian, únicamente, por el color en el que se encuentra un determinado texto. Sin embargo, el mayor problema viene cuando estas limitaciones pueden llegar a arruinar el juego, como ocurre en \textit{The Witness}, un juego de puzzles mencionado en apartados anteriores en el que, para poder solventar muchos de dichos puzzles, requiere que el usuario distinga (o al menos tenga la información), distintos colores. 
También nos encontramos con juegos de lucha en dos dimensiones donde la única diferencia entre los escenarios del fondo y los personajes principales es el color de los mismos. No ver la diferencia complica a que el jugador detecte si un elemento está al frente o al fondo de dicho escenario.

\subsection{Cómo solventar parte de estos problemas para los daltónicos}
\label{sec:daltonicossolventar}

Para poder dar respuesta a esta cuestión, decidí preguntar en \textit{Reddit} a los usuarios daltónicos sobre cómo desarrollar un juego que sea adecuado para ellos \footnote{\url{https://goo.gl/d6cTqe}}. Fueron muchas las respuestas obtenidas, pero todo se puede resumir en un par de ideas.

La primera es que, en la medida de lo posible, nunca tengamos que diferenciar dos cosas distintas simplemente por el color. Tal y como un usuario comentó, si estamos desarrollando un juego naval y la diferencia entre un barco que está ``sano'' y uno que está ``roto'' es que cambiamos el color o la silueta del barco de verde a rojo, también podemos añadir chispas u otros elementos a mayores que faciliten la idea de que algo ha cambiado y ayude al usuario a apreciarlo visualmente con elementos a mayores. Del mismo modo, si tenemos un juego como \textit{Tetris} o \textit{Candy Crush} donde las piezas son relevantes, en vez de distinguirlas por color, podríamos hacerlas de diferentes formas o añadir una imagen a cada una de ellas.

La segunda idea es que, si es completamente necesario introducir diferentes colores para diferenciar ciertos elementos, o bien usamos colores que, generalmente, no crean ningún problema como el azul, amarillo, verde... (lo cual no nos garantiza que hayamos solventado el problema, dado que hay muchas formas de daltonismo y en algunas de ellas el usuario todavía puede tener problema diferenciándolos dependiendo del color contreto usado) o creamos una opción en el que podamos cambiar la paleta de colores usada. Hay algunos videojuegos (la mayor parte de ellos independientes), donde se da esta opción. 

\begin{figure}[H]
		\includegraphics[width=\textwidth,height=\textheight,keepaspectratio]{./img/redditcolorblind1.png}
	\caption{Consejo para ayudar a los daltónicos a la hora de desarrollar un juego}
	\label{fig:roomsgamecolorblind1}
\end{figure}

\begin{figure}[H]
		\includegraphics[width=\textwidth,height=\textheight,keepaspectratio]{./img/redditcolorblind2.png}
	\caption{Consejos y ejemplos para evitar que las personas daltónicas tengan problemas}
	\label{fig:roomsgamecolorblind2}
\end{figure}

\begin{figure}[H]
		\includegraphics[width=\textwidth,height=\textheight,keepaspectratio]{./img/redditcolorblind3.png}
	\caption{Un usuario nos comenta la combinación de colores que son problemáticas}
	\label{fig:roomsgamecolorblind3}
\end{figure}

\subsection{Cómo hemos solventado el problema en nuestro proyecto}
\label{sec:solventadodaltonicos}

En nuestro caso, al haber preguntado de antemano a los usuarios, siempre tuvimos desde el primer momento la idea de crear la interfaz gráfica con soporte para daltónicos en mente, a pesar de que, al haber tenido en cuenta a las personas invidentes, cualquier persona podría jugarlo sin ninguna dificultad.

Todos los elementos que se encuentran en la pantalla están diferenciados con caracteres completamente diferentes por lo que, incluso aunque todos los colores fueran iguales, sería sencillo identificar cada elemento. 

También hemos creado una opción que cambia la paleta de colores a utilizar y facilita su visualización para aquellos usuarios con este inconveniente.


        \chapter{Fundamentos Tecnológicos}

En este capítulo hablaremos sobre los fundamentos tecnológicos que vamos a usar en este proyecto y, si cabe, la razón por la que fueron elegidos. En primer lugar citaremos las herramientas que hemos usado y, en segundo lugar, las bibliotecas que hemos decidido utilizar.

\section{Herramientas empleadas}

\paragraph{Java} Lenguaje de programación orientado a objetos cuya primera aparición fue en 1995. Es uno de los lenguajes de programación más utilizados en la industria y una de sus principales características es que es multiplataforma, es decir, puede ser ejecutado en cualquier sistema operativo que tenga la \textit{Java Virtual Machine} instalada sin necesidad de realizar cambios en el código (WORA\footnote{\textit{Write once, run anywhere}. Eslogan creado por Sun Microsystems para mostrar los beneficios de la multiplataforma}). Esta ventaja es primordial en nuestro caso, dado que la mayoría de las personas que pueden estar interesadas en el proyecto usan una gran variedad de sistemas operativos.

 \paragraph{Eclipse} Es un IDE\footnote{\textit{Integrated Development Environment}. Entorno de desarrollo integrado} usado para escribir código en múltiples idiomas. También incluye una serie de \textit{plugins} que facilitan y automatizan muchas de las labores a realizar como el uso de sistema de controles, ejecución de código y tests, herramientas de \textit{debug}, autocompletado de código, etc.

 \paragraph{Git} Sistema de control de versiones distribuido introducido en 2005 y desarrollado principalmente por Linus Torvalds. Es el control de versiones referencia en la mayoría de empresas y proyectos de software libre gracias a su rapidez y, al ser distribuido, permite trabajar y realizar \textit{commits} del código sin necesidad de conexión a internet.

 \paragraph{GitHub} Plataforma de desarrollo colaborativo usada para alojar proyectos usando el sistema de control de versiones Git. La mayoría de proyectos de código abierto lo usan, dado que es gratuito, aunque también tiene la opción de almacenar el código de forma privada tras, previamente, realizar un pago.

 \paragraph{Listas de Correo} Las listas de correo son un método de comunicación muy usado por diferentes comunidades, especialmente en el desarrollo de software, que ayudan a los usuarios que participan en ellas a enviar correos a múltiples personas que lo deseen de forma anónima y, al mismo tiempo, tener un historial de las respuestas dadas por los mismos. En nuestro caso la hemos usado para comunicarnos con un grupo de usuarios y desarrolladores de videojuegos para invidentes y recibir \textit{feedback} por parte de la comunidad.

 \paragraph{Reddit} Web creada en 2005 y que actualmente se encuentra en el top 50 de las más visitadas del mundo. Cuenta con una comunidad gigante que está dividida en muchísimos subgrupos dependiendo del tema a tratar. La hemos usado como una herramienta de \textit{feedback}. Especialmente los \textit{subreddits} de \href{https://www.reddit.com/r/ColorBlind/}{daltónicos}, de \href{https://www.reddit.com/r/blind/}{personas que sufren de ceguera}, \href{https://www.reddit.com/r/gamedev/}{desarrolladores de videojuegos} y \href{https://www.reddit.com/r/roguelikes/}{roguelikes}.

\paragraph{Dia} Aplicación informática que permite la creación de todo tipo de diagramas. En nuestro caso lo hemos usado para crear los diagramas \textit{UML} que se encuentran en esta memoria.

\paragraph{Gantt Project} Programa de software libre diseñado para la creación de diagramas de Gantt.

\paragraph{JSON} \textit{JavaScript Object Notation}. Es un formato muy usado en APIs para intercambio de datos, similar a XML. En nuestro caso lo usamos para definir las gramáticas y diccionarios de nuestro proyecto, dado que es muy sencillo de leer y especificar. Hay numerosas bibliotecas que nos permiten analizar y trabajar con este formato en Java. La que nosotros usamos es \textit{Gson}.

\paragraph{\LaTeX} Sistema de composición de textos altamente usado por la mayoría de textos científicos dada la facilidad de su composición, simpleza, alta calidad y herramientas que a ayudan a la creación de fórmulas, inserción de imágenes y muchos otros elementos. Muy modificable. Es el sistema que hemos usado para la creación de este documento.

\paragraph{NVDA} Lector de pantalla de código libre para Windows. Orca es, en cierta medida, su equivalente en Linux.

\section{Bibliotecas empleadas}

\paragraph{Gson} Biblioteca usada para transformar archivos JSON a objetos de Java y viceversa.

\paragraph{JCurses} JCurses\footnote{\textit{The Java Curses Library}} es una biblioteca para el desarrollo de aplicaciones de terminal para JAVA. Es similar a AWT\footnote{\textit{Abstract Window Toolkit.} Kit de herramientas de interfaz de usuario de la plataforma original de Java}, pero basada en el sistema de ventanas \textit{Curses} de \textit{UNIX}.

\paragraph{Libjcsi} Biblioteca de representación gráfica que trabaja sobre JCurses y simplifica la tarea de representar y refrescar elementos del terminal.

        \chapter{Metodología}

En este apartado describiremos la metodología llevada a cabo en el proyecto. Una metodología es un conjunto de procesos, métodos y prácticas llevadas a cabo para asegurar, en la mayor medida posible, calidad en el producto final y en el tiempo acordado.

En nuestro caso, al ser un proyecto realizado por una sola persona y con un tiempo diario muy limitado, hemos optado por adaptar una serie de ideas y valores principales de varias metodologías.

\section{Desarrollo en cascada}

\section{Scrum}

\subsection{Prácticas recomendadas de Scrum}

\subsection{Valores de Scrum}

\subsubsection{Concentración}


\subsubsection{Coraje}


\subsubsection{Compromiso}


\subsubsection{Sinceridad}


\subsubsection{Respeto}

\section{Otras consideraciones}

\section{Metodoloxía seguida}


% Para a realización deste proxecto usarase unha mestura de varias metodoloxías,

% procesos e ideas de métodos de desarrollo de software. Nunha vista máis xeral do

% proxecto usaremos desenvolvemento en cascada. Isto é, para cada unha das

% características a implementar primero analizaremos os seus requisitos, a continuación

% procederemos a deseñalo e, por último, programalo e verificalo. Isto farase de manera

% global (para o análise e deseño teremos en conta o proxecto en xeral), pero cada unha

% destas características consta dun número concreto de tareas individuales que

% deberemos de programar e verificar individualmente. Para facilitar este traballo

% usaremos "desenrolo guiado por probas" (Test Driven Development), é dicir, para cada

% tarea, en vez de comezar a programala nada máis podamos, primeiro faremos as probas

% (unit tests ou functional tests, dependendo do que sea necesario) basándose nos

% requisitos especificados nos primeros pasos da cascada (análise e deseño) e logo a

% programaremos. Deste xeito estaremos seguros de que o que queremos cumplirase tal e

% como o especificamos e teremos o test como proba delo.

        \chapter{Planificación y Seguimiento}

En este capítulo detallaramos la planificación de cada una de las iteraciones que hemos hecho, así como el seguimiento realizado. La elaboración de este proyecto se llevó a cabo durante dos años, comenzando en junio de 2014 y teniendo un par de parones durante varios meses (desde septiembre de 2014 hasta enero de 2015 y desde mayo de 2015 hasta septiembre del mismo año). 
Por lo tanto, los periodos de actividad serían los siguientes:

\begin{itemize}
  \item Desde junio del año 2014 hasta septiembre del 2014 (3 meses).
  \item Desde enero de 2015 hasta mayo del mismo año (4 meses).
  \item Desde septiembre del 2015 hasta abril de 2016 (8 meses).
\end{itemize}

En las próximas secciones hablaremos de lo que hemos desarrollado durante estos tres periodos de tiempo y las iteraciones seguidas en los mismos.

Cabe destacar que todos los datos y tareas mostradas a continuación están realizadas de forma lineal, dado que solamente disponemos de un recurso.

\section{Junio 2014 - Septiembre 2014}

Este periodo comienza el 4 de junio, que es cuando se nos comenta la posibilidad de realizar este proyecto, y termina el 24 de septiembre, por lo que consta de un total de 15 semanas. Hay que tener en cuenta que durante parte de este verano (desde agosto) es cuando me mudé a Holanda y empiecé mi \textit{internship}, por lo que las jornadas en días laborables solamente constaban de una o dos horas y alrededor de 8/9 horas durante los fines de semana.

Sumando todo el tiempo empleado durante estas semanas se calcula que se han empleado 190 horas en total.

Durante este tiempo nos centramos en recabar información general para poder empezar el proyecto con buen pie, así como la implementación de los mapas y habitaciones.

\subsection{Primera iteración: Análisis de requisitos generales, diseño genérico y preparación y configuración de los elementos necesarios para el comienzo de la implementación}

Desde el 4 de junio hasta el 29 de junio.

\paragraph{Análisis de requisitos generales} Al desarrollar un proyecto enfocado a una parte de la población de la que no formas parte, es muy importante documentarse sobre todos los aspectos que hay que tener en cuenta e intentar ponerse en su piel (por ejemplo usar las herramientas que utilizan diariamente para recabar ideas).
Àsí mismo, desarrollar un videojuego puede llegar a ser una tarea infinita, dado que nuevas características o ideas que añadir es algo que sucede casi diaremente. Aprender sobre lo básico del género y poner límites es fundamental para centrar los pocos recursos que tenemos en crear lo completamente necesario.

\paragraph{Diseño genérico del juego a implementar} Crearemos el primer diseño genérico que nos dará una idea sobre lo que tendremos que realizar y nos guiará sobre el proceso de creación del juego. Este primer boceto cambiará a medida que querramos añadir nuevos elementos y querramos especificar más otros.

\paragraph{Búsqueda de bibliotecas que se adapten a nuestros requisitos} Hay varias bibliotecas con las que se puede crear una interfaz gráfica sencilla como la de Rogue, pero todas ellas tienen sus ventajas e inconvenientes. Debemos de averiguar cuál de ellas es la más adecuada para nuestro caso.

\paragraph{Creación y configuración del ambiente de desarrollo para poder empezar la implementación} Al empezar un nuevo proyecto debemos de crear un repositorio en git, instalar todo el software necesario y preparar lo básico para que podamos empezar a programar sin encontrarnos con ninguna dificultad a posteriori.

\subsubsection{Tareas y seguimiento}

La descomposición de las tareas es la siguiente:

\begin{enumerate}[label=\bfseries WBS 1.\arabic*]
  \item Análisis de requisitos generales
    \begin{enumerate}[label=\bfseries WBS 1.1.\arabic*]
      \item Estudio de herramientas para invidentes.
      \item Estudio de los elementos del género \textit{roguelike}.
      \item Analizar los elementos encontrados y especificación de lo que queremos en nuestro caso.
    \end{enumerate}
  \item Diseño genérico del juego a implementar.
  \item Búsqueda de bibliotecas que se adapten a nuestros requisitos.
  \item Creación y configuración del ambiente de desarrollo para poder empezar la implementación.
\end{enumerate}

Para la realización de todas estas tareas se planificaron 65 horas en total. Esta estimación se cumplió, por lo que al final de la iteración todas las tareas habían sido realizadas. Ver la figura ~\ref{fig:sec1it1}

\begin{figure}
    \includegraphics[width=\textwidth,height=\textheight,keepaspectratio]{./img/sec1it1.png}
  \caption{Diagrama de Gantt de la primera iteración de la primera etapa}
  \label{fig:sec1it1}
\end{figure}

\subsubsection{Qué hemos conseguido en esta iteración}

Comenzar un nuevo proyecto y recabar toda la información necesaria para tener una idea donde asentar las bases en las que se sembrarán las próximas iteraciones siempre es una tarea ardua y complicada. En esta iteración hemos sentado estas bases, definiendo lo que debemos de realizar y creando un primer boceto de lo que tendremos que cumplir durante los próximos meses.

\subsubsection{Qué queremos conseguir en la próxima iteración}

Con todo lo básico definido, ahora debemos de materalizarlo. Durante las siguientes iteraciones deberemos de empezar a crear el juego en sí, empezando por los mapas y habitaciones.

\subsection{Segunda iteración: Generación de mapas y habitaciones}

Desde el 15 de julio hasta el 24 de septiembre.

\paragraph{Análisis de requisitos de los mapas y habitaciones} Hay mucho escrito y realizado sobre la creación de mapas y habitaciones de forma aleatoria. En base toda esta información, debemos de elegir lo que queremos realizar en base a, por ejemplo, si el tamaño de dicho mapa siempre será el mismo o no, cómo y cuántas habitaciones queremos tener en cada mapa y el tipo de aletoriedad que queremos (completamente aleatorio o pseudo-aleatorio).

\paragraph{Diseño de los mapas y las habitaciones} Una vez hayamos creado el análisis de requisitos y tengamos toda la información necesaria sobre la mesa, es hora de crear el diseño. Dicho diseño debe de ser fácilmente extendible para que la realización de pequeños cambios no sean un gran reto.

\paragraph{Creación de tests que cubran lo analizado} Tal y como comentamos anteriormente, antes de ponernos con la programación, crearemos los tests que especifiquen y cumplan lo diseñado para, posteriormente, empezar con la codificación.

\paragraph{Implementación del diseño de mapas y habitaciones} Con el diseño y los tests creados, es hora de la implementación del código analizado y diseñado.

\subsubsection{Tareas y seguimiento}

La descomposición de las tareas es la siguiente:

\begin{enumerate}[label=\bfseries WBS 2.\arabic*]
  \item Análisis de requisitos de los mapas y habitaciones.
    \begin{enumerate}[label=\bfseries WBS 2.1.\arabic*]
      \item Estudio de diferentes algoritmos de creación aleatoria de mapas y habitaciones.
      \item Decisión sobre la estructura y tamaño a elegir en base al tipo del juego.
    \end{enumerate}
  \item Diseño de los mapas y las habitaciones.
  \item Creación de tests que cubran lo analizado.
  \item Implementación del diseño de mapas y habitaciones.
\end{enumerate}

Para la realización de esta segunda iteración se planificaron 125 horas en total. La estimación no fue correcta y no fuimos capaces de finalizar todo lo necesario, por lo que tuvimos que dejar la tarea de representar este mapa para la siguiente iteración. Ver la figura ~\ref{fig:sec1it2}

\begin{figure}
    \includegraphics[width=\textwidth,height=\textheight,keepaspectratio]{./img/sec1it2.png}
  \caption{Diagrama de Gantt de la segunda iteración de la primera etapa}
  \label{fig:sec1it2}
\end{figure}

\subsubsection{Qué hemos conseguido en esta iteración}

Hemos conseguido generar mapas y habitaciones de manera aleatoria, pero todavía no mostrarlos en la interfaz de usuario. Uno de los puntos esenciales del género de los \textit{roguelike} es su aleatoriedad y conseguir que los mapas es el primer gran punto.

\subsubsection{Qué queremos conseguir en la próxima iteración}

En la siguiente iteración debemos de terminar lo que no hemos conseguido hacer en ésta, por lo que representar estos mapas en la interfaz gráfica toma prioridad. A continuación comenzaremos con la creación de objetos. Dichos objetos es otro de los puntos esenciales del género.

\section{Enero 2015 - Mayo 2015}

Este periodo comienza el 2 de enero, tras los días libres de navidad y nuevo año, y termina el 31 de mayo, en el que pararemos para preparar los exámenes. Por lo tanto, este periodo consta de un total de 21 semanas. Durante estas semanas he estado trabajando a tiempo completo y algunos días se los dediqué a estudiar para los exámenes, por lo que el tiempo semanal empleado fue de alrededor de 12 horas de media, aunque algunas de estas semanas fueron de más trabajo que otras.

En total, 170 horas fueron dedicadas para este sprint inicialmente.

Durante este sprint nos hemos centrado en continuar lo realizado anteriormente y, en general, seguir con el desarrollo del juego en sí.

\subsection{Primera iteración: mapas en IU, análisis, diseño, tests e implementación de los objetos}

Esta primera iteración comienza el 2 de enero y termina el 4 de marzo.

\paragraph{Mostrar los mapas y habitaciones en el interfaz de usuario} En la anterior fase implementamos los mapas, pero no los enseñamos en la interfaz de usuario. En esta ocasión debemos de acabar con esta tarea para terminar con todo lo relacionado con la creación de mapas.

\paragraph{Análisis de requisitos para sobre los objetos} En un \textit{roguelike} los objetos son primordiales. Debemos de investigar qué objetos vamos a usar y cómo los representaremos en el mapa diseñado anteriormente.

\paragraph{Crear un diseño simple sobre cómo los objetos interactuarán con el mapa} Debemos de crear un diseño genérico que nos permita añadir objetos de manera sencilla.

\paragraph{Creación de tests que cubran lo analizado} Al igual que antes, es necesario crear primero los tests en vez de empezar con la implementación del diseño directamente. En este caso también añadiremos tests que interactuen mapas y objetos para asegurarnos que no vamos a tener ningún error cuando combinemos ambos.

\paragraph{Implementación de los objetos en el juego} Una vez realizado el análisis, el diseño y los tests, podremos implementar la solución encontrada.

\subsubsection{Tareas y seguimiento}

La descomposición de las tareas es la siguiente:

\begin{enumerate}[label=\bfseries WBS 1.\arabic*]
  \item Mostrar los mapas y habitaciones en el interfaz de usuario.
  \item Análisis de requisitos sobre los objetos.
    \begin{enumerate}[label=\bfseries WBS 1.1.\arabic*]
      \item Buscar información sobre los diferentes tipos de objetos necesarios en el juego.
      \item Estudiar cómo estos objetos deben de interactuar con el mapa.
      \item Decidir y resumir lo encontrado.
    \end{enumerate}
  \item Crear un diseño simple sobre cómo los objetos interactuarán con el mapa.
  \item Creación de tests que cubran lo analizado.
  \item Implementación de los objetos en el juego y su asociación con el propio mapa y habitaciones.
\end{enumerate}

Para la realización de la primera iteración del segundo bloque de trabajo se planificaron 55 horas en total. La estimación fue correcta y fue posible cumplirla. Ver la figura ~\ref{fig:sec2it1}

\begin{figure}
    \includegraphics[width=\textwidth,height=\textheight,keepaspectratio]{./img/sec2it1.png}
  \caption{Diagrama de Gantt de la primera iteración de la segunda etapa}
  \label{fig:sec2it1}
\end{figure}

\subsubsection{Qué hemos conseguido en esta iteración}

Terminar la tarea restante del sprint anterior y la creación de diferentes objetos (pociones, armas y armaduras) que los personajes podrán utilizar. También podemos representar dichos objetos en el mapa, además de que la creación de nuevos elementos sea muy sencilla.

\subsubsection{Qué queremos conseguir en la próxima iteración}

Crear los personajes. Ésta es uno de los últimos pilares esenciales del género, por lo que tomará prioridad y será realizado a continuación.

\subsection{Segunda iteración: análisis, diseño, tests e implementación de los personajes}

La segunda iteración comienza el 5 de marzo y acaba el 23 de mayo.

\paragraph{Análisis de requisitos para sobre personajes} En el juego tendremos un personaje principal (controlado por el usuario) y una serie de enemigos con diferentes características que intentarán destruirle y que el usuario deberá de batir. En esta tarea nos encargaremos de buscar información sobre el tipo de enemigos a crear y diferentes métodos para hacerlo de forma más genérica posible, dado que ser capaz de crear estos enemigos fácilmente es una característica fundamental.

\paragraph{Creación del diseño de los personajes} Una vez hayamos realizado el análisis para obtener la idea de los enemigos y personajes a usar, es hora de crear el diseño. Tal y como hemos comentado anteriormente, es necesario hacer un diseño extendible donde crear diferentes clases de enemigos sea lo más sencillo posible.

\paragraph{Implementación de los tests} Como hasta ahora, antes de empezar con la implementación directamente, debemos de crear tests sobre ello, que nos indicará las características que deben de cumplir.

\paragraph{Programación de lo diseñado y analizado previamente} Con el análisis, diseño y tests listos, podremos realizar la tarea de implementación.

\subsubsection{Tareas y seguimiento}

La descomposición de las tareas es la siguiente:

\begin{enumerate}[label=\bfseries WBS 2.\arabic*]
  \item Análisis de requisitos sobre personajes (jugadores y enemigos)
    \begin{enumerate}[label=\bfseries WBS 2.1.\arabic*]
      \item Buscar información sobre los diferentes tipos de enemigos a los que nos enfrentaremos.
      \item Estudiar cómo estos enemigos interectuarán con el mapa y los objetos creados anteriormente.
      \item Decidir y resumir lo encontrado.
    \end{enumerate}
  \item Crear un diseño para crear, fácilmente, nuevos enemigos que aparezcan aleatoriamente en el mapa.
  \item Creación de tests que cubran lo analizado.
  \item Implementación de los enemigos en el juego y su asociación con el propio mapa, habitaciones y objetos.
\end{enumerate}

Para la realización de este \textit{sprint} se han asignado 80 horas, las cuales fueron suficientes para cubrir todo lo que se quiso realizar. Ver la figura ~\ref{fig:sec2it2}

\begin{figure}
    \includegraphics[width=\textwidth,height=\textheight,keepaspectratio]{./img/sec2it2.png}
  \caption{Diagrama de Gantt de la segunda iteración de la segunda etapa}
  \label{fig:sec2it2}
\end{figure}

\subsubsection{Qué hemos conseguido en esta iteración}

Al igual que con los objetos, hemos conseguido facilitar su creación, así como la representación del mapa de los mismos. Hemos creado un personaje principal (controlado por el usuario) y diferentes tipos de enemigos (doblins, ratas y dragones). Ambos tipos se muestran en el mapa.

\subsubsection{Qué queremos conseguir en la próxima iteración}

Tal y como tenemos el proyecto actualmente, tenemos el mapa creado aleatoriamente y podemos mostrar objetos y personajes con facilidad. El siguiente paso consiste en que los personajes puedan interactuar con los objetos.

\subsection{Tercera iteración: interacción entre personajes, tests e implementación}

Esta última iteración de la sección, y la más corta, comienza el 24 de mayo y acaba el 31 del mismo mes.

En la anterior iteración teníamos el objetivo de crear los personajes y enemigos y que éstos fueran capaces de ser representados en la interfaz gráfica. En este caso daremos un paso más allá y deberemos de añadir funciones que dichos personajes puedan realizar con los objetos: creación de un inventario para que puedan almacernarlos y, de tal modo, el usuario y enemigos puedan tener la habilidad de coger objetos (del mapa al inventario), tirarlos (del inventario al mapa), equiparlos (del inventario al personaje) y desequiparlos (del personaje al inventario).

\paragraph{Ampliar los tests sobre la interacción entre los objetos y personajes} En la iteración anterior fuimos capaces de crear los personajes de forma genérica y en este caso deberemos crear nuevas funciones que ayuden a la interacción entre los objetos y los personajes, tal y como hemos descrito anteriormente.

\paragraph{Implementación en base a los tests, análisis y diseño de la anterior iteración} El diseño y el análisis ya lo tenemos hecho y, una vez tengamos los tests creados, podremos realizar la implementación.

\subsubsection{Tareas y seguimiento}

La descomposición de las tareas es la siguiente:

\begin{enumerate}[label=\bfseries WBS 3.\arabic*]
  \item Ampliar los tests que tengan que ver con la interacción entre los objetos y personajes
  \item Implementación en base al diseño de la anterior iteración y los nuevos tests creados.
\end{enumerate}

\begin{figure}
    \includegraphics[width=\textwidth,height=\textheight,keepaspectratio]{./img/sec2it3.png}
  \caption{Diagrama de Gantt de la tercera iteración de la segunda etapa}
  \label{fig:sec2it3}
\end{figure}

Al ser un pequeño sprint, hemos sido capaces de terminarlo sin problema. El diagrama de Gantt puede verse en la figura ~\ref{fig:sec2it3}

\subsubsection{Qué hemos conseguido en esta iteración}

Interacción básica entre personajes y objetos.

\subsubsection{Qué queremos conseguir en la próxima iteración}

Aumentar esta interacción (que podamos atacar a los personaje, por ejemplo) y dar control al usuario, por lo que tendremos que diseñar la manera de que el usuario pueda decidir moverse, atacar, coger un objeto del mapa, etc.

\section{Septiembre 2015 - Abril 2016}

Este periodo empieza el 1 de septiembre, tras un breve descanso en verano, y termina el 31 de abril. Durante este tiempo hemos empleado más tiempo que anteriormente en el proyecto y desarrollado mucho más (20 horas a la semana). También hemos buscado \textit{feedback} una vez tuvimos tiempo disponible para poder tener tiempo para implementarlo y asegurarnos de que lo que hemos realizado es aceptado por la comunidad. 
Este periodo consta de un total de 34 semanas.

En total, 680 horas fueron dedicadas para este sprint inicialmente.

\subsection{Primera iteración: Aumentar interacción entre objetos, mapas y personajes, añadir movimiento para el jugador}

Esta primera iteración da comienzo el 1 de septiembre y acaba el 5 de octubre, es decir, damos 5 semanas para su realización.

\paragraph{Aumentar la interacción entre objetos, personajes y mapa} Hasta el momento tenemos el mapa, objetos y personajes, pero debemos de ser capaces de interactuar completamente y aumentar en características. Por ejemplo añadiendo puertas entre habitaciones, que también se muestren en el mapa, que los personajes puedan atacarse entre sí (teniendo en cuenta las armaduras y armas que llevan) y la creación de un campo de visión para el usuario para que solamente pueda ver lo que hay a su alrededor y no todo el mapa.

\paragraph{Agregar \textit{listeners} para que el jugador pueda decidir lo que hacer} En este momento el jugador no puede hacer nada. Ahora es el momento de introducir los elementos necesarios para que nos podamos mover por el mapa y realizar todas las acciones implementadas anteriormente con nuestro personaje.

\subsubsection{Tareas y seguimiento}

La descomposición de las tareas es la siguiente:

\begin{enumerate}[label=\bfseries WBS 1.\arabic*]
  \item Aumentar la interacción entre objetos, personajes y mapa.
    \begin{enumerate}[label=\bfseries WBS 1.1.\arabic*]
      \item Crear tests para los elementos posteriores
      \item Añadir puertas que unan las habitaciones para que el jugador pueda pasar de habitación a habitación.
      \item Añadir la posibilidad de ataque entre personajes.
      \item Añadir un campo de visión al usuario para que no sea capaz de ver todo el mapa, pero solamente cierta área a su alrededor.
    \end{enumerate}
  \item Agregar \textit{listeners} para que el jugador pueda decidir lo que hacer.
\end{enumerate}

Hemos asignado 100 horas para la estimación de este \textit{sprint}, las cuales cumplimos sin problema. Ver la figura ~\ref{fig:sec3it1}

\begin{figure}
    \includegraphics[width=\textwidth,height=\textheight,keepaspectratio]{./img/sec3it1.png}
  \caption{Diagrama de Gantt de la primera iteración de la tercera etapa}
  \label{fig:sec3it1}
\end{figure}

\subsubsection{Qué hemos conseguido en esta iteración}

Al acabar esta iteración tenemos lo básico de un juego: nos podemos mover por un mapa, coger objetos y combatir enemigos (aunque estos enemigos todavía no tienen inteligencia artificial, por lo que no se mueven).

\subsubsection{Qué queremos conseguir en la próxima iteración}

En la próxima iteración deberemos terminar lo básico del juego. Para ello tendremos que implementar el sistema de portales (que nos transportarán a un mapa completamente diferente, consiguiendo puntos de esta manera), añadir básica inteligencia artifial a los enemigos e incorporar la opción de cambiar el color de la interfaz de usuario para que se adapte a los jugadores daltónicos, tal y como hemos explicado en la sección \label{sec:daltonicossolventar}

\subsection{Segunda iteración: Diseño e implementación de los portales, IA de los enemigos y accesibilidad para daltónicos}

Esta segunda iteración comienza el 6 de octubre y termina el 8 de noviembre, con 4 semanas y media para su finalización.

\paragraph{Diseño e implementación de los portales} El sistema de portales fue la idea principal que tuvimos para crear un sistema de puntuación y progreso para el usuario. El objetivo está en encontrar el portal dentro del mapa y, una vez encontrado, un nuevo mapa será generado, igual que otro portal en ese mapa. De esta manera la meta del juego se centra en derrotar enemigos (consiguiendo mejores armas y armaduras en el proceso) para encontrar el mayor número de portales posibles.
Deberemos de diseñar e implementar estos portales en esta iteración.

\paragraph{Añadir inteligencia artificial} Dependiendo del enemigo al que nos enfrentemos, éste se comportará de forma distinta. Habrá enemigos que sean activos y vayan contra el usuario, mientras que otros serán pasivos y no quieran atacar al usuario. Puede ser que en el futuro queramos incrementar el número de este tipo de comportamientos, por lo que deberemos de diseñarlo de la mejor forma.

\paragraph{Añadir opciones para cambiar el color de la interfaz gráfica} El punto central de este proyecto es la accesibilidad e incluir una opción para que daltónicos puedan disfrutar de nuestro juego es decisivo.

\subsubsection{Tareas y seguimiento}

La descomposición de las tareas es la siguiente:

\begin{enumerate}[label=\bfseries WBS 2.\arabic*]
  \item Diseño e implementación de los portales.
    \begin{enumerate}[label=\bfseries WBS 2.1.\arabic*]
      \item Diseñar la mejor manera para incluir los portales en nuestro diseño (en el análisis es algo que se había considerado hacer).
      \item Creación de los tests.
      \item Implementación de los portales.
    \end{enumerate}
  \item Añadir inteligencia artificial.
  	\begin{enumerate}[label=\bfseries WBS 2.2.\arabic*]
      \item Diseñar la mejor manera para incluir diferentes tipos de IA.
      \item Crear los tests de IA.
      \item Implementar esta IA para los enemigos.
    \end{enumerate}
  \item Añadir opciones para cambiar el color de la interfaz gráfica para usuarios daltónicos
\end{enumerate}

48 horas fueron las que creímos suficientes para la estimación del \textit{sprint} y que fueron necesarias para la conclusión del mismo con todas las tareas terminadas. Ver la figura ~\ref{fig:sec3it2} como referencia.

\begin{figure}
    \includegraphics[width=\textwidth,height=\textheight,keepaspectratio]{./img/sec3it2.png}
  \caption{Diagrama de Gantt de la segunda iteración de la tercera etapa}
  \label{fig:sec3it2}
\end{figure}

\subsubsection{Qué hemos conseguido en esta iteración}

Con esta última iteración tenemos la parte básica del juego completada. Hay un objetivo, enemigos con IA, diferentes objetos, puertas que conectan distintas habitaciones y varias acciones que se pueden realizar.

\subsubsection{Qué queremos conseguir en la próxima iteración}

Con el juego en sí completado (o por lo menos la parte primordial), es hora de empezar a analizar y diseñar la parte de la generación automática del lenguaje.

\subsection{Tercera iteración: Análisis, diseño básico y comienzo de la implementación de las gramáticas}

Esta segunda iteración comienza el 9 de noviembre y termina el 31 de diciembre, con 6 semanas y media para su finalización.

\paragraph{Análisis de las gramáticas} Una parte muy importante del juego es crear frases que sean generadas de forma automática y pseudo-aleatoria en base a una serie de gramáticas y diccionario dado. Si conseguimos realizar esto, crear diferentes idiomas sería trivial (solamente tendríamos que cambiar la gramática en caso de que en dicho idioma sea diferente y traducir el diccionario). Para ello tendremos que investigar cómo podemos hacer esto en nuestro caso y cómo han solventado este problema otra gente.

\paragraph{Diseño general sobre la generación del lenguaje} Una vez hemos analizado y decidido lo que debemos de realizar, crearemos un diseño lo más simple y adaptable posible para la generación de dichas frases en base a las gramáticas dadas.

\paragraph{Creación de gramáticas y diccionarios base} Para empezar el desarrollo necesitamos tener una gramática y diccionaro base con lo que poder ver los diferentes resultados obtenidos.

\subsubsection{Tareas y seguimiento}

La descomposición de las tareas es la siguiente:

\begin{enumerate}[label=\bfseries WBS 3.\arabic*]
  \item Análisis de las gramáticas.
    \begin{enumerate}[label=\bfseries WBS 3.1.\arabic*]
      \item Leer la información que existe sobre la generación automática de lenguaje.
      \item Estudiar cómo lo podemos implementar en nuestro proyecto con lo ya existente.
      \item Tomar la decisión en base a lo encontrado y sentar las bases sobre ello.
    \end{enumerate}
  \item Diseño general sobre la generación del lenguaje y cómo las gramáticas interactuarán con nuestro programa.
  \item Creación de gramáticas y diccionarios base para tener una base con lo que testar lo que implementaremos.
\end{enumerate}

Hemos reservado 130 horas para esta iteración y hemos conseguido terminar todas las tareas asignadas a tiempo. Ver la figura ~\ref{fig:sec3it3} para más información.

\begin{figure}
    \includegraphics[width=\textwidth,height=\textheight,keepaspectratio]{./img/sec3it3.png}
  \caption{Diagrama de Gantt de la tercera iteración de la tercera etapa}
  \label{fig:sec3it3}
\end{figure}

\subsubsection{Qué hemos conseguido en esta iteración}

Hemos sentado las bases de lo que queremos implementar con las gramáticas.

\subsubsection{Qué queremos conseguir en la próxima iteración}

Empezar con los tests y desarrollo de lo investigado en esta iteración.

\subsection{Cuarta iteración: Análisis, diseño y comienzo de la implementación de las gramáticas}

Esta segunda iteración comienza el 1 de enero y termina el 14 de enero. Es decir, tiene 2 semanas.

\paragraph{Análisis sobre cómo crear las gramáticas NP} Las gramáticas NP\footnote{Sintagma nominal.} son básicas, pero no iguales en todos los idiomas. Debemos de estudiar qué planteamientos existen que faciliten su implementación de la mejor forma posible.

\paragraph{Diseño de las gramáticas NP} Las frases de sintagma nominal serán las más usadas dentro de nuestro juego. No solamente serán utilizadas para la generación de las frases, pero también a la hora de representar los elementos en la interfaz de usuario. Por ello debemos de crearlas de la forma más extendible y accesible que podamos.

\paragraph{Creación de los tests para las gramáticas NP} Al igual que en otras iteraciones, realizamos los tests antes de nada.

\paragraph{Implementación de las gramáticas NP} Con los tests realizados, debemos de empezar con la implementación. Tenemos que tener en cuenta que hay que ser capaces de sustituir las clases de palabras por las palabras en sí que se encuentra en el diccionario. También hay que tener en cuenta las restricciones posibles (por el momento solamente contamos con las restricciones de inglés. Es decir, número).

\subsubsection{Tareas y seguimiento}

La descomposición de las tareas es la siguiente:

\begin{enumerate}[label=\bfseries WBS 4.\arabic*]
  \item Análisis sobre cómo crear las gramáticas NP de forma genérica y fácilmente extendibles para otros idiomas.
  \item Diseño de las gramáticas NP.
  \item Creación de los tests para las gramáticas NP.
  \item Implementación de las gramáticas NP y \textit{link} con el diccionario.
\end{enumerate}

Hemos reservado 40 horas para esta iteración y terminamos todas las tareas a tiempo. Ver la figura ~\ref{fig:sec3it4} para más información.

\begin{figure}
    \includegraphics[width=\textwidth,height=\textheight,keepaspectratio]{./img/sec3it4.png}
  \caption{Diagrama de Gantt de la cuarta iteración de la tercera etapa}
  \label{fig:sec3it4}
\end{figure}

\subsubsection{Qué hemos conseguido en esta iteración}

Al finalizar esta iteración hemos conseguido generar sintagmas nominales de forma pseudo-aleatoria que basan su información en las gramáticas y diccionarios dados.

\subsubsection{Qué queremos conseguir en la próxima iteración}

Aumentar estas gramáticas para que lo generado no solamente sean sintagmas nominales, pero también frases que usen estos sintagmas nominales.

\subsection{Quinta iteración: Análisis, diseño e implementación de las gramáticas y frases más complejas}

Esta segunda iteración comienza el 15 de enero y termina el 29 de enero. Es decir, tiene 2 semanas y media de duración, igual que la iteración anterior.

\paragraph{Análisis sobre cómo crear las gramáticas complejas de forma genérica} Una vez somos capaces de crear sintagmas nominales, deberemos de ser capaces de combinarlos con otros elementos del lenguaje como verbos para crear frases que sean capaces de describir todo lo que realmente deseamos. Deberemos de analizar las diferentes formas de hacer esto.

\paragraph{Diseño de dichas gramáticas} Una vez hemos analizado y decidido los pasos que vamos a seguir, nos quedará realizar el diseño del mismo.

\paragraph{Creación de los tests para las gramáticas complejas} Al igual que en las veces anteriores, tendremos que crear los tests antes de empezar con la implementación.

\paragraph{Implementación y \textit{link} con el diccionario} Implementar las gramáticas más complejas que se comuniquen con las gramáticas NP y nos permitan generar frases que describan lo que ocurre en el juego de forma aleatoria y automática.

\subsubsection{Tareas y seguimiento}

La descomposición de las tareas es la siguiente:

\begin{enumerate}[label=\bfseries WBS 5.\arabic*]
  \item Análisis sobre cómo crear las gramáticas complejas de forma genérica, extendibles para otros idiomas y que hagan uso de las NP.
  \item Diseño de dichas gramáticas.
  \item Creación de los tests para las gramáticas complejas, al igual que su relación con las NP y diccionario.
  \item Implementación y \textit{link} con el diccionario.
\end{enumerate}

Hemos reservado 48 horas para esta iteración y la hemos acabado a tiempo. En la figura ~\ref{fig:sec3it5} se muestra el diagrama de Gantt de esta iteración.

\begin{figure}
    \includegraphics[width=\textwidth,height=\textheight,keepaspectratio]{./img/sec3it5.png}
  \caption{Diagrama de Gantt de la quinta iteración de la tercera etapa}
  \label{fig:sec3it5}
\end{figure}

\subsubsection{Qué hemos conseguido en esta iteración}

Tener gramáticas que generen frases de manera pseudo-aleatoria para todas las descripciones que contemplamos actualmente en el juego.

\subsubsection{Qué queremos conseguir en la próxima iteración}

Las frases son generadas, pero todavía no las mostramos en ninguna parte de la interfaz gráfica, por lo que el jugador no puede ni verla ni escucharla. En el siguiente sprint tenemos que añadir esta opción, además de añadir otros idiomas (es decir, añadir las gramáticas y diccionarios) como el gallego y español, lo que significará añadir las restricciones adicionales que tienen de género.

\subsection{Sexta iteración: Implementación de las restricciones, gallego y castellano, creación de la interfaz de usuario para con las frases generadas y primer vídeo}

La sexta iteración dará comienzo el 30 de enero y terminará el 6 de febrero. Es decir, tiene solamente 1 semana de desarrollo, pero fue una semana de 8 horas al día en la práctica, por lo que en total fueron 56 horas de trabajo en la misma.

\paragraph{Creación de la interfaz de usuario para mostrar las frases generadas al usuario} Hasta ahora somos capaces de generar frases que describan lo que sucede en el juego, pero no las mostramos. Ahora es el momento de mostrar en una ventana la frase generada.

\paragraph{Implementación de las restricciones para el resto de idiomas} En inglés solamente tenemos la restrición en número, pero otros idiomas, como el castellano o gallego, tienen un restricciones adicionales como el género. Tenemos que tener esto en cuenta e introducirlo en el código. 

\paragraph{Añadir gramáticas y diccionarios para gallego y castellano} La idea desde un principio fue en tener el proyecto en la mayor cantidad posible de idiomas, por lo que añadirlo en gallego y castellano es importante.

\paragraph{Crear primer vídeo para recibir \textit{feedback}} Al terminar esta iteración tendremos la base del proyecto completada. Por ello crearemos un vídeo mostrando el punto en el que estamos y con el que recibiremos comentarios con lo que podremos crear mejoras en las futuras iteraciones \footnote{\url{https://www.youtube.com/watch?v=RgND1IGZ-68}}.

\paragraph{Añadir las teclas necesarias para mostrar las descripciones del inventario y ambiente} El usuario podrá pedir que se genere una frase que describa algo en concreto. Se debería de realizar en base a lo que está especificado en el diagrama de casos de uso.

\subsubsection{Tareas y seguimiento}

La descomposición de las tareas es la siguiente:

\begin{enumerate}[label=\bfseries WBS 6.\arabic*]
  \item Creación de la interfaz de usuario para mostrar las frases generadas al usuario.
  \item Añadir las teclas necesarias para mostrar las descripciones del inventario y ambiente, tal y como está definido en el diagrama de casos de uso.
  \item Implementación de las restricciones para el resto de idiomas.
  \item Añadir gramáticas y diccionarios para gallego y castellano.
  \item Crear primer vídeo para recibir \textit{feedback}.
\end{enumerate}

En la figura ~\ref{fig:sec3it6} se muestra el diagrama de Gantt de esta iteración. Hemos reservado, tal y como ya hemos comentado, 56 horas, que fueron suficiente para terminar todas las tareas mencionadas.

\begin{figure}
    \includegraphics[width=\textwidth,height=\textheight,keepaspectratio]{./img/sec3it6.png}
  \caption{Diagrama de Gantt de la sexta iteración de la tercera etapa}
  \label{fig:sec3it6}
\end{figure}

\subsubsection{Qué hemos conseguido en esta iteración}

Terminar lo básico del juego, generando las frases necesarias al usuario y añadiendo más idiomas disponibles.

\subsubsection{Qué queremos conseguir en la próxima iteración}

Debemos de esperar por los comentarios recibidos en el vídeo y, mientras, podremos mejorar diferentes aspectos y características del juego haciendo las tareas que hay en el \textit{backlog}.

\subsection{Séptima iteración: Resolución de bugs detectados y añadidas pequeñas funcionalidades}

La séptima iteración empieza el 7 de febrero y acaba el 21 de febrero, con una duración de 2 semanas, volviendo a las 20 horas de trabajo por semana y 40 horas en total para la iteración.

\paragraph{Solucionar bug donde la pantalla no se refresca} En algunas ocasiones y al realizar ciertas acciones, la interfaz gráfica no se actualiza hasta la siguiente acción. Tenemos que solucionar este problema para que lo que se muestre siempre sea lo más nuevo.

\paragraph{Cambiar las teclas que usamos por defecto} Las teclas que tenemos por defecto no son del todo intuitivas. Deberemos de cambiarlo para que sí lo sean.

\paragraph{Hacer los adjetivos que definen los personajes variables} Si un enemigo tiene poca vida y el personaje que controla el usuario tiene mucha más vida, habrá la posibilidad de que el adjetivo a usar sobre el enemigo sea ``pequeño'', ``insignificante''... mientras que si es al revés, se usará ``grande'', ``poderoso'', de forma que se verá al enemigo de forma diferente.

\paragraph{Añadir opción para que las descripciones sean descriptivas o numéricas} Cuando escuchamos una descripción, a veces queremos que dicha descripción sea numérica, es decir, que mencione exactamente la vida o posición de los personajes. Otras veces deseamos que esto no sea así y que todas las descripciones se hagan de una forma más poética y se usen palabras para las definiciones. Crearemos una opción en la que el usuario será capaz de seleccionar lo que prefiera.

\subsubsection{Tareas y seguimiento}

La descomposición de las tareas es la siguiente:

\begin{enumerate}[label=\bfseries WBS 7.\arabic*]
  \item Solucionar bug donde la pantalla no se refresca cuando es necesario.
  \item Cambiar las teclas que usamos por defecto.
  \item Cambiar los adjetivos que definen los personajes dependiendo de la vida de dichos personajes.
  \item Añadir opción para que las descripciones sean descriptivas o numéricas, dependiendo de lo que el usuario desee.
\end{enumerate}


En la figura ~\ref{fig:sec3it7} se muestra el diagrama de Gantt de esta iteración.

\begin{figure}
    \includegraphics[width=\textwidth,height=\textheight,keepaspectratio]{./img/sec3it7.png}
  \caption{Diagrama de Gantt de la séptima iteración de la tercera etapa}
  \label{fig:sec3it7}
\end{figure}

\subsubsection{Qué hemos conseguido en esta iteración}

Mejorar el juego en diferentes aspectos al añadir nuevas funcionalidades que afectarán directamente al jugador y solucionar algunos de los \textit{bugs} que detectados mientras jugábamos.

\subsubsection{Qué queremos conseguir en la próxima iteración}

El día 19 de febrero es cuando recibimos el feedback que presentamos en la anterior iteración, así que crearemos un sprint donde podamos analizarla e implementar los cambios que nos piden.

\subsection{Octava iteración: Implementación en base al \textit{feedback} recibido}

En esta iteración implementaremos los cambios más importantes sobre el \textit{feedback} recibido. Dado que estos comentarios pueden llevar bastante tiempo (quizás haya más cambios o los realizados desencadenan más ideas a implementar), es mejor terminarlos en menor tiempo posible para poder recibir \textit{feedback} más rápidamente.
Por este motivo esta iteración contará solamente con 4 días y 20 horas en total, empezando el 22 y acabando el 26, que es cuando crearemos el vídeo.

\subsubsection{Feedback recibido}

Todo este feedback es el recibido por parte de mis supervisores que a su vez se basaron en lo que ellos mismos y otros usuarios les han comentado.

\paragraph{Tener en cuenta la persistencia del tiempo} Si matamos a un enemigo, sería una buen adición que las descripciones lo tuvieran en cuenta. Por ejemplo a la hora de enumerar los elementos que hay alrededor del usuario, se podría comentar que también se encuentra un ``goblin'' muerto.

\paragraph{Cambiar el sistema de salida de las frases generadas} Hasta ahora, cada vez que una frase era generada (tanto a petición del usuario o en base a algo que ha sucedido en el juego), mostrábamos una nueva ventana con dicha frase, que tendríamos que cerrar en cada ocasión. Esto es un inconveniente para aquellas personas que sí pueden ver y no quieren ser molestados por este tipo de ventanas y para los usuarios invidentes tampoco es la mejor solución disponible, dado que no es lo que se suelen encontrar en otros juegos del género. Lo que mayoría de juegos usan es una \textit{textarea}, es decir, una ventana aparte donde se vaya almacenando todas las frases generadas, de tal forma que siempre podemos volver a ella cuando queramos y servirá como un log. Lo importante está en cambiar el ``foco'' del juego a esta ventana cada vez que una frase sea generada para que el lector sea capaz de leerla.

\paragraph{Pequeños cambios en los adjetivos usados} En algunas ocasiones usábamos adjetivos poco comunes a la hora de definir ciertos nombres.

\paragraph{Adición de niveles y experiencia} Los enemigos, armas y el propio usuario deberían de tener niveles para que el juego escale en dificultad. Cada vez que se destruye un enemigo, el personaje del usuario ganará una serie de puntos que serán usados para subir niveles.

\paragraph{Cambio en la aleatoriedad} Hasta ahora, todo lo generado era aleatorio (el tipo de enemigos que nos encontrábamos, los objetos que soltaban, los propios objetos encontrados...). Es mejor que esta aleatoriedad venga dada por el nivel del usuario para que el juego escale mejor en dificultad. Esto se explica con mayor profundidad en la sección: TODOOOOOOOOOOOOOOOOOOOOOOO

\paragraph{Evitar la repetitividad} Algunas de las frases que generamos tienden a ser bastante repetitivas (el héroe desequipa la espada, el héroe desequipa la armadura...), en vez de tener un lenguaje más natural (el héroe desequipa la espada, la armadura...). Esto sucede en un par de casos y debemos de tratarlo.

\paragraph{Darle nombre al héroe} Siempre nos referimos al personaje que controla el usuario como ``héroe'' o por su pronombre (``él''). Podríamos darle un nombre o dejar que el usuario elija para dar una mayor variedad.

\paragraph{Mostrar una serie de estadísticas al cambiar de mazmorra} Cuando pasamos de una mazmorra a otra usando el portal, podríamos mostrar una serie de estadísticas de los enemigos batidos, la cantidad de experiencia obtenida, el nivel actual del usuario...

A mayores, crearemos otro vídeo informando sobre los cambios que hemos realizado en base a estos comentarios \footnote{\url{https://www.youtube.com/watch?v=3lS0WFrwOeQ}}.

\subsubsection{Tareas y seguimiento}

La descomposición de las tareas que realizaremos este \textit{sprint} es la siguiente:

\begin{enumerate}[label=\bfseries WBS 8.\arabic*]
  \item Tener en cuenta la persistencia del tiempo.
  \item Cambiar el sistema de salida de las frases generadas.
  \item Pequeños cambios en los adjetivos usados.
  \item Adición de niveles y experiencia.
  \item Cambio en la aleatoriedad.
  \item Evitar la repetitividad.
\end{enumerate}

En la figura ~\ref{fig:sec3it8} se muestra el diagrama de Gantt de esta iteración.

\begin{figure}
    \includegraphics[width=\textwidth,height=\textheight,keepaspectratio]{./img/sec3it8.png}
  \caption{Diagrama de Gantt de la octava iteración de la tercera etapa}
  \label{fig:sec3it8}
\end{figure}

\subsubsection{Qué hemos conseguido en esta iteración}

Hemos conseguido implementar los cambios más importantes en base a los comentarios recibidos.

\subsubsection{Qué queremos conseguir en la próxima iteración}

Seguir implementando el resto de cambios y tener en cuenta el \textit{feedback} del resto de usuarios.
        \chapter[Análisis de Requisitos]{Análisis de Requisitos globales}

En este capítulo explicaremos el proceso de análisis de requisitos llevado a cabo para la elaboración de la aplicación y toda la información recibida por la comunidad, así como las decisiones tomadas en base a la experiencia de otros proyectos similares al nuestro.

\section{Consultas con la comunidad}

A la hora de realizar las consultas con la comunidad decidimos contactar al inicio del proyecto con la sede local de la ONCE para recabar información sobre las dificultades que los invidentes se encuentran a la hora de utilizar diferentes dispositivos y programas informáticos.
También hemos preguntado a personas daltónicas para que nos aconsejarán desde su experiencia sobre cómo se deberían realizar juegos sin que ellos se vean afectados por su diseño.

Ambas consultas, así como el resultado obtenido de las mismas, se detallaron en la Sección~\ref{sec:dificultadesfeedback} 

\subsection{Resumen de las peticiones recibidas}
	\paragraph{Posibilidad de cambiar el color de la interfaz} La interfaz gráfica tendrá diferentes colores que diferencien los distintos tipos de enemigos y elementos del juego de cara a los jugadores videntes. Tal y como se menciona en la Sección~\ref{sec:solventadodaltonicos}, debemos tener en cuenta específicamente el caso de los usuarios daltónicos a pesar de que dichos elementos son fácilmente diferenciables.
	\paragraph{Variedad en las descripciones automáticas de los elementos del juego} Las personas invidentes necesitan una forma alternativa y no visual de saber qué es lo que está sucediendo en el juego y así poder tomar una decisión razonada en base a la situación en la que se encuentran. Para ello el sistema deberá generar una descripción en el lenguaje natural de lo que ocurre dentro del juego, para que el software lector pueda comunicárselo al jugador. Debemos generar estas descripciones de la forma más variada y correcta posible para que no sean repetitivas. 
	\paragraph{Diferentes idiomas} Al ser un videojuego en el cual el lenguaje es esencial, debemos tener en cuenta la posibilidad de incluir otros idiomas y así ampliar el conjunto de usuarios potenciales.
  \paragraph{Utilizar el lenguaje a nuestro favor} Usar elementos de temporalidad o diferentes adjetivos para definir ciertos elementos para aumentar la expresividad del sistema y que el lenguaje sea parte de la experiencia, consiguiendo asó una mayor inmersión del usuario en el juego.
	\paragraph{Multiplataforma} Tal y como mencionamos en la introducción, nuestro software debe poder ser ejecutado en varios sistemas operativos: Linux, Mac OS, Windows, etc.
	
\subsection{Cómo hemos abordado el problema de accesibilidad para invidentes en nuestro proyecto}

En el proyecto hemos creado descripciones para todos los elementos del juego visibles en la pantalla, por lo que el jugador siempre puede saber dónde se encuentra, qué hay a su alrededor, cuáles son sus características, las del enemigo y las de los elementos equipables, etc. También hemos creado una ventana que se encuentra al lado de la pantalla de juego clásica y donde se van guardando en orden inverso todas las descripciones generadas, siendo la primera de ellas la última generada y leyéndose automáticamente por el reproductor de pantalla que esté siendo usado en ese momento. De esta forma, una persona con visión siempre puede leer dicha frase mientras que el jugador invidente puede volver a escucharla las veces que quiera. Además, dicho listado funciona a modo de \textit{log}, por lo que el jugador también puede recordar lo sucedido anteriormente gracias a que estos mensajes permanecen disponibles en la pantalla y pueden ser leídos por el software lector. La Figura \ref{fig:roomsgametextarea} muestra la captura de pantalla de esta ventana para una sesión de juego.

\begin{figure}[H]
		\includegraphics[width=\textwidth,height=\textheight,keepaspectratio]{./img/roomsGameTextArea.png}
	\caption{Captura de pantalla del área donde mostramos las frases generadas por nuestro juego para invidentes}
	\label{fig:roomsgametextarea}
\end{figure}

\subsection{Cómo hemos abordado el problema de accesibilidad para daltónicos en nuestro proyecto}
\label{sec:solventadodaltonicos}

En nuestro caso, al haber preguntado de antemano a los potenciales usuarios, siempre tuvimos la idea de crear la interfaz gráfica con soporte para daltónicos en mente.

De este modo, todos los elementos del juego que se encuentran en la interfaz gráfica son distinguibles entre sí gracias al uso de caracteres completamente diferentes por lo que, incluso aunque todos los colores fueran iguales, sería sencillo identificar cada elemento sólo por su forma en vez de por su color. 

De todas maneras, para aquellos usuarios que sufren daltonismo hemos creado una opción que cambia la paleta de colores a utilizar y facilita su visualización.

\section{Análisis de los elementos del juego}
Al ser un \textit{roguelike}, debemos incluir los elementos característicos del género: aleatoriedad, dificultad, progreso, etc. Los detalles acerca de los mismos, así como la forma en la que han sido introducidos en el proyecto han sido ya comentados en la Sección~\ref{sec:roguelikeinformacion}.

\section{Requisitos del aplicativo}
Con toda la información obtenida y analizada, creamos una lista con los casos de uso que nuestro proyecto debe cumplir y cuyo diagrama mostramos en la Figura \ref{fig:roomsgametextarea}:

\begin{itemize}
  \item \textbf{Movimiento} Un usuario siempre será capaz de moverse con su personaje a una posición válida dentro de la habitación donde se encuentra. 
  \item \textbf{Ataque normal} Cuando el personaje está en la misma posición que un enemigo, éste será capaz de atacar cuerpo a cuerpo.
  \item \textbf{Ataque mágico} Cuando el personaje está dentro de una distancia determinada de un enemigo, aquél será capaz de realizar un ataque mágico, siempre y cuando tenga sufienciente maná (energía mágica) para el mismo. Si el enemigo se encuentra demasiado lejos, el ataque mágico se realizará, pero sin afectar a ningún enemigo (por lo que el usuario simplemente perderá ese maná).
  \item \textbf{Coger elemento} En las habitaciones del mapa habrá elementos que se pueden recolectar, así como enemigos que soltarán diferentes objetos.
  \item \textbf{Equipar elemento} Poder equipar a nuestro personaje con un objeto que está en el inventario, siempre y cuando no tengamos un objeto del mismo tipo ya equipado.
  \item \textbf{Desequipar elemento} De la misma manera, podremos desequipar un objeto que tenemos equipado mientras tengamos espacio en el inventario.
  \item \textbf{Tirar elemento} En algunas ocasiones el jugador podría encontrarse con que no dispone de espacio suficiente en el inventario para almacenar nuevos elementos, por lo que podrá arrojar objetos al suelo para hacer hueco a aquéllos nuevos que queramos recoger.
  \item \textbf{Descripciones} Durante el transcurso del juego podremos generar diferentes descripciones de lo que ocurre en el juego dependiendo de lo que queramos saber. Por ejemplo: lo que el personaje del jugador tiene en el inventario, las posiciones a las que nos podemos mover, la descripciones de los enemigos a los que nos enfrentamos, las estadísticas del héroe y de los enemigos, etc.
  \item \textbf{Activación de descripciones numéricas} Algunas de estas descripciones corresponden a posiciones o características que el usuario podría querer escuchar, bien como valor numérico (por ejemplo la cantidad de vida de un monstruo) o bien mediante expresiones del lenguaje (``mucha'', ``poca'', ``bastante'', etc.), dependiendo de lo que prefiera.
  \item \textbf{Cambio de colores} Para usuarios daltónicos hemos incluido una opción para cambiar los colores de la interfaz gráfica.
\end{itemize}

\noindent Asimismo, algunas de estas acciones consumirán un turno de juego al realizarse: coger, equipar, desequipar y tirar un objeto, moverse y atacar. Al consumir el jugador su turno, pasará entonces el turno a los enemigos que, a su vez, realizarán sus acciones como, por ejemplo, acercarse o atacar al jugador.

\begin{figure}[h!]
    \centering
    \includegraphics[width=0.9\textheight,angle=90]{img/casosdeuso.png}
    \caption{Diagrama UML de casos de uso del proyecto}
    \label{fig:casosdeuso}
\end{figure}
        \chapter{Diseño e Implementación}

En este capítulo mostraremos los detalles del dise;o y la implementacion de diferentes partes del proyecto.


        \chapter{Recepción, conclusión y trabajo futuro}

En este último capítulo detallaremos la recepción y el \textit{feedback} recibido tras mostrar nuestro producto a la comunidad, la conclusión sacadas de la elaboración del proyecto y el posible trabajo futuro del mismo.

\section{Recepción y \textit{Feedback}}

TODO:

\section{Conclusión}

Realizar un proyecto de estas características no es sencillo. Dejando a un lado la parte técnica y de diseño, que nunca es trivial, en este caso se requiere un gran grado de investigación, análisis, asimilación del \textit{feedback} y determinación para poder sacarlo adelante, pero durante su desarrollo siempre me he encontrado a una comunidad muy ilusionada por su existencia y encantada de colaborar y resolver mis dudas. 
Es una pena que algo tan esencial como son la mayoría de elementos tecnológicos de hoy en día (y su \textit{software} en particular) todavía no estén adaptados para que toda la parte de la población pueda usarlos y sean dejados de lado, muchas veces por mera ignorancia. Espero que con este proyecto haya gente que se anime a intentar hacer algo similar (o mejorar lo ya creado) y que entre todos podamos formar una sociedad donde no haya tanta discriminación tecnológica.

\section{Trabajo Futuro}

Cuando se crea cualquier tipo de \textit{software}, siempre existen mejoras y nuevas funcionalidades que puede ser añadidas para mejorar lo creado, sobre todo si se trata de un juego de \textit{software libre}, dado que las comunidades de desarrolladores y jugadores suelen ser bastante apasionadas. Esta es la razón por la que, desde un principio, hemos definido lo que queremos hacer como básico para luego incluir otros elementos que no son tan esenciales.

Durante el desarrollo del proyecto se han tenido muchas ideas que hemos ido recopilando e implementando en caso de que las considerásemos necesarias. Actualmente todas estas ideas se encuentran en la página del proyecto en Github\footnote{\url{https://github.com/dpenas/roomsgame/issues}}, donde cualquier persona puede añadir su comentario o incluso realizar el cambio necesario y crear una \textit{pull request} para que sea incluido en el juego final.

Algunas de estas ideas con las que podemos mejorar el juego en el futuro son las siguientes: 

\paragraph{Creación de un menú:} En la actualidad no tenemos un menú porque podemos realizar todo lo necesario dentro del propio juego. Sin embargo, y a medida que la cantidad de opciones crece, sería conveniente su creación para que nos mostrase lo que podemos cambiar sin tener que entrar en el juego en sí.

\paragraph{Añadir más variedad de enemigos y elementos:} La cantidad de enemigos que tenemos no es muy grande y el juego se beneficiaría teniendo más enemigos y objetos con los que interactuar, sin olvidar que deben de escalar de manera apropiada.

\paragraph{Crear un modo historia:} El juego actual es libre. Esto quiere decir que las partidas son independientes entre sí y no hay un principio y final definido, sino que el usuario es el que crea, en cierta manera, su historia. 
No estaría de más tener un modo de juego diferente que lleve al usuario a ciertos niveles y pantallas predefinidas para aquellos usuarios que prefieran este tipo de aventuras.

\paragraph{Crear un tutorial:} La información sobre cómo empezar a jugar se encuentra disponible en la página de Github del proyecto, pero no estaría de más la creación de un tutorial o pantalla de inicio que explicase al usuario qué hacer y cómo desenvolverse por los entornos del juego.

\paragraph{Crear un ``modo resumen'':} Con todas las frases que generamos durante la partida podríamos crear una especie de historia que relatase, en la medida de lo posible, lo que el jugador ha hecho durante dicha partida, dándole cierta importancia a los ``hitos'' conseguidos durante la misma.

\paragraph{Incluir elementos sonoros:} Algunos juegos para invidentes solamente tienen elementos auditivos para informar al usuario sobre lo que tiene a su alrededor. En nuestro caso, al poder crear frases en base a unas gramáticas dadas, no resulta ser algo necesario, pero sí que podríamos crear una opción para que, en vez de usar siempre descripciones, podamos reproducir sonidos a diferentes grados de volumen para informar al jugador sobre distintos elementos.

\paragraph{Añadir más complejidad a las gramáticas:} Las frases creadas se basan en gramáticas dadas, pero hay algunas particularidades (como frases reflexivas), que son más complejas de crear. Podríamos introducir esta opción para que la cantidad de frases generadas y su variedad sea todavía mayor. 

\paragraph{Añadir más complejidad a los diccionarios:} Tener un diccionario propio es necesario, dado que no todos los idiomas disponen del mismo tipo de bibliotecas y su uso limitaría cantidad de idiomas con los que podremos jugar. Sin embargo, si el juego está en inglés o español, sí que podríamos usar alguna biblioteca para que, en vez de usar la palabra que se encuentra en nuestro diccionario, busque un sinónimos en dichas bibliotecas, lo que aumentaría de manera significante la variedad de las palabras que usamos. 

\paragraph{Crear opción de jugar en base a una semilla:} Algunos \textit{roguelikes} o \textit{roguelites} como \textit{The Binding of Isaac: Rebirth} \footnote{\url{http://store.steampowered.com/app/250900}}, permiten que el jugador introduzca un código que equivale a una semilla de aleatoriedad para que los elementos generados sean en base a esa semilla. Esto permitiría que los usuarios compartan partidas donde todo lo generado sea igual para ambos.

        \appendix
        \chapter{Escoger la licencia}

Cuando tenemos que escoger una licencia para un nuevo programa también debemos tener en cuenta las licencias de las bibliotecas y recursos que hemos usado en el mismo. Las bibliotecas que hemos usado en nuestro proyecto y sus respectivas licencias son éstas:

\begin{itemize}
  \item \textbf{Gson} Apache 2.0 license
  \item \textbf{JCurses} GNU Lesser General Public License v3
  \item \textbf{libjcsi} GNU Lesser General Public License v2.1
\end{itemize}

\noindent Como podemos apreciar, Gson usa una licencia Apache 2.0\cite{website:apachelicense} mientras que JCurses y libjcsi tienen GNU Lesser GPL \cite{website:gnulicense}.
La licencia de Apache 2.0 no es compatible con las versiones de GNU Lesser GPL anteriores a la 3, pero las versiones de GNU Lesser GPL son compatibles entre sí.
Por este motivo decidimos usar GNU General Public Licence v3, dado que es compatible con todas las licencias de las librerías que hemos usado.

        \chapter{Instalación e Instrucciones}





\end{document}

