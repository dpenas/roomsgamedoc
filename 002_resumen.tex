\section*{Resumen:}

La industria del entretenimiento digital ha crecido inmensamente en los últimos años, llegando a alcanzar números de ventas jamás vistos anteriormente.
Parte de la razón de este crecimiento viene dada por una mejora radical en el aspecto visual, necesaria para que el jugador se sienta inmerso en la aventura que se le está planteando.
Estas mejoras, sin embargo, dejan de lado a muchos jugadores que, por diferentes motivos, no son capaces de apreciar el contenido visual que se les ofrece o tienen problemas para ello, haciendo imposible que disfruten del contenido ofrecido.

Este proyecto consiste en la creación de ``The accessible dungeon'', un videojuego de género \textit{roguelike} para invidentes que, desde un principio, parte de la idea de generar contenido específicamente diseñado para que pueda ser jugado por todo el mundo, haciendo énfasis en ofrecer al jugador una diversa cantidad de frases que describan lo que está sucediendo en su alrededor y que serán generadas automáticamente en base a una serie de gramáticas y diccionarios, empleando para ello técnicas de Procesamiento del Lenguaje Natural.
