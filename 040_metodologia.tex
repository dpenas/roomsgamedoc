\chapter{Metodología}

En este apartado describiremos la metodología llevada a cabo en el proyecto. Una metodología es un conjunto de procesos, métodos y prácticas llevadas a cabo para asegurar, en la mayor medida posible, calidad en el producto final y en el tiempo acordado.

En nuestro caso, al ser un proyecto realizado por una sola persona y con un tiempo diario muy limitado, hemos optado por adaptar una serie de ideas y valores principales de varias metodologías.

\section{Desarrollo en cascada}

\section{Scrum}

\subsection{Prácticas recomendadas de Scrum}

\subsection{Valores de Scrum}

\subsubsection{Concentración}


\subsubsection{Coraje}


\subsubsection{Compromiso}


\subsubsection{Sinceridad}


\subsubsection{Respeto}

\section{Otras consideraciones}

\section{Metodoloxía seguida}


% Para a realización deste proxecto usarase unha mestura de varias metodoloxías,

% procesos e ideas de métodos de desarrollo de software. Nunha vista máis xeral do

% proxecto usaremos desenvolvemento en cascada. Isto é, para cada unha das

% características a implementar primero analizaremos os seus requisitos, a continuación

% procederemos a deseñalo e, por último, programalo e verificalo. Isto farase de manera

% global (para o análise e deseño teremos en conta o proxecto en xeral), pero cada unha

% destas características consta dun número concreto de tareas individuales que

% deberemos de programar e verificar individualmente. Para facilitar este traballo

% usaremos "desenrolo guiado por probas" (Test Driven Development), é dicir, para cada

% tarea, en vez de comezar a programala nada máis podamos, primeiro faremos as probas

% (unit tests ou functional tests, dependendo do que sea necesario) basándose nos

% requisitos especificados nos primeros pasos da cascada (análise e deseño) e logo a

% programaremos. Deste xeito estaremos seguros de que o que queremos cumplirase tal e

% como o especificamos e teremos o test como proba delo.
