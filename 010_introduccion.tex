\chapter[Introdución]{Introducción}

En este capítulo introductorio se explicarán los aspectos necesarios para entender lo más importante del proyecto, la motivación para la realización del mismo y un breve resumen del resto de capítulos que forman parte de la memoria.

\section{Videojuegos y personas invidentes}
La mayor parte de los videojuegos comerciales no tienen en cuenta a muchas minorías de la sociedad. Haciendo una pequeña búsqueda online pueden encontrarse miles de usuarios quejándose de \textit{first person shooters. Género de videojuegos en primera persona donde el usuario dispara con diferentes armas a enemigos.} que tienen un \textit{campo de visión} limitado, que termina causándoles mareos al poco rato; usuarios zurdos que tienen que acomodarse a ciertos controles a no existir una opción para cambiarlos; usuarios daltónicos protestando que diferentes juegos (como por ejemplo The Witness\footnote{Juego de puzzles en primera persona: \url{http://the-witness.net/}}), basan buena parte de su mecánica en que el jugador sea capaz de distinguir diferentes colores; usuarios invidentes que no pueden disfrutar de prácticamente ninguno de los títulos que se encuentran en el mercado, etc.

En este proyecto nuestro objetivo ha sido el crear un videojuego desde cero que tenga en cuenta la problemática de las personas con deficiencias visuales, tomando especial relevancia el desarrollo para invidentes y centrándose en los aspectos que sean relevantes para ellos.

\subsection{Visita a la \textit{ONCE}}
El pasado curso, junto a uno de mis co-directores, asistí de tiflotecnología organizado por la Organización Nacional de Ciegos Españoles (ONCE) donde, entre otros temas, uno de sus formadores nos habló sobre la tiflotecnología, nos mostró la forma en la que usaban ordenadores y teléfonos móviles, incluso para leer código fuente y los errores, muchos de ellos fácilmente solventables, que se cometían día a día en temas de accesibilidad por parte de los desarrolladores. Esta visita nos resultó muy interesante y gracias a ella fuimos capaces de detectar posibles mejoras que tener en cuenta en el proyecto y los fallos que no deberíamos cometer en su diseño. También se ofrecieron a probar el proyecto una vez estuviera listo, pero este tema será tratado en las próximas secciones.

\section{Motivación}
A nivel personal, como la mayor parte de las personas de mi generación actual, el entretenimiento digital siempre ha sido parte de mi vida y, en mi caso, una de las razones por las que desde pequeño estuve interesado en la informática y motivo por el que, finalmente, acabé estudiando esta carrera. Del mismo modo, y habiéndome relacionado con bastante gente con toda clase de necesidades especiales durante un gran periodo de tiempo, mi interés por la tiflotecnología y las limitaciones que la tecnología ofrece a millones de personas no hizo más que crecer año a año.

\section{Estructura de la memoria}

La memoria está formada por ocho capítulos en los que se explican los pasos tomados a la hora de crear el proyecto.

\paragraph*{Capítulo 1. Introducción.}
El presente capítulo, donde se explican, de manera general y resumida, en qué consiste el proyecto, la motivación del mismo y la estructura de la memoria.

\paragraph*{Capítulo 2. Estado del Arte.}
En este apartado hablaremos de otros proyectos similares, de la situación actual de la industria sobre el problema que tratamos en este proyecto y del diferente software que es utilizado en relación con la tiflotecnología.

\paragraph*{Capítulo 3. Fundamentos Tecnológicos.}
Trataremos las herramientas y recursos empleados durante la elaboración del proyecto.

\paragraph*{Capítulo 4. Metodología.}
Se detallarán las prácticas y metodologías de desarrollo empleadas y la razón de su uso.

\paragraph*{Capítulo 5. Planificación y Seguimento.}
Detallaremos la planificación y seguimiento de cada una de las etapas del proyecto. 

\paragraph*{Capítulo 6. Análisis de Requisitos.}
Se comentará el análisis de requisitos para este proyecto y se explicará en detalle cada uno de los mismos.

\paragraph*{Capítulo 7. Diseño e Implementación.}
En este capítulo explicaremos los detalles del diseño e implementación del sistema, centrándonos especialmente en aquellas partes del programa que consideramos más relevantes.

\paragraph*{Capítulo 8. Conclusión y trabajo futuro.}
En este apartado se comentarán las conclusiones obtenidas y se detallarán los posibles cambios, mejoras y extras que se
podrían tener en cuenta en el futuro para su mejora.