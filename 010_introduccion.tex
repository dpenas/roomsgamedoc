\chapter[Introdución]{Introdución}

En este capítulo introductorio se explicarán los aspectos necesarios para entender lo más importante del proyecto, la motivación para realización del mismo y un breve resumen del resto de capítulos que forman parte de la memoria.

\section{Videojuegos y personas invidentes}
La mayor parte de los videojuegos comerciales no tienen en cuenta a muchas minorías de la sociedad. Haciendo una pequeña búsqueda online pueden encontrarse miles de personas quejándose de \textit{first person shooters} que tienen un \textit{FOV}\footnote{\textit{Field of view}, campo de visión. Extensión de mundo observable en un momento dado} limitado, causándoles mareos al poco rato; daltónicos protestando que diferentes juegos (como por ejemplo The Witness\footnote{Juego de puzzles en primera persona: \url{http://the-witness.net/}}), basan buena parte de su mecánica en que el jugador sea capaz de distinguir diferentes colores; zurdos que tienen que acomodarse a ciertos controles a no existir una opción para cambiarlos; invidentes que no pueden disfrutar de prácticamente ninguno de los títulos que se encuentran en el mercado, etc.

En este proyecto nuestro objetivo es el de crear un videojuego desde cero que tenga en cuenta todo tipo de minorías, tomando especial relevancia el desarrollo para invidentes y centrándose en los aspectos que sean relevantes para ellos; descripciones que sean fácilmente reproducibles que cambien automáticamente y fácil expansión del título, tanto en características generales como en idiomas, gramáticas o palabras empleadas en el mismo.

\subsection{Visita a la \textit{ONCE}}
A principios de 2015 Jesús y yo fuimos a un taller de la ONCE donde una persona invidente nos habló sobre la tiflotecnología, nos mostró la forma en la que usaban ordenadores y móviles, incluso para leer código y los errores, muchos de ellos fácilmente solventables, que se cometían día a día en temas de accesibilidad por parte de diversas compañías. Esta visita nos resultó muy interesante y gracias a ella fuimos capaces de detectar posibles mejoras que tener en cuenta en el proyecto y los fallos que no deberíamos de cometer en su diseño. También se ofrecieron a probar el proyecto una vez estuviera listo, pero este tema será tratado en las próximas secciones.

\section{Motivación}
El entretenimiento digital siempre ha sido parte de mi vida y una de las razones por las que desde pequeño estuve interesado en la informática y motivo por el que, finalmente, acabé estudiando esta carrera. Del mismo modo, y habiéndome relacionado con bastante gente con toda clase de necesidades especiales durante un gran periodo de tiempo, mi interés por la tiflotecnología y las limitaciones que la tecnología ofrece a millones de personas no hizo más que crecer año a año.

A pesar de los grandes avances de la industria de los videojuegos y de la gran cantidad de nuevos estudios y proyectos que se lanzan anualmente, resulta muy complicado poder dedicarse profesionalmente a cualquiera de estas dos cosas (y no digamos ambas a la vez) por lo que en mi futuro profesional no he sido capaz, al menos de momento, de cumplir mi sueño de trabajar en lo que más me apasiona. Por este motivo, cuando Jesús me comentó que él y Carlos llevaban un tiempo con este proyecto disponible, no dudé en un instante en aceptarlo y ponerme manos a la obra.

Este videojuego se ha desarrollado durante el periodo de un par de años en los que se incluyen muchos cambios en mi vida, tales como mi estancia permanente en Holanda hace ya casi dos años y mis primeros pasos en el mundo laboral. Cada día que pasa me alegro más de tener un proyecto como éste, dado que sin la pasión e interés por el mismo estoy seguro de que jamás lo habría terminado.

\section{Estructura de la memoria}

La memoria está formada por ocho capítulos en los que se explican los pasos tomados a la hora de crear el proyecto.

\paragraph*{Capítulo 1. Introducción.}
Se explicarán, de manera general y resumida, en qué consiste el proyecto, la motivación del mismo y la estructura que tendrá la memoria.

\paragraph*{Capítulo 2. Estado del arte.}
En este apartado hablaremos de otros proyectos similares, de la situación actual de la industria sobre el problema que tratamos en este proyecto y del diferente software que es utilizado en relación con la tiflotecnología.

\paragraph*{Capítulo 3. Fundamentos Tecnológicos.}
Citaremos y hablaremos sobre las herramientas y bibliotecas empleadas durante la elaboración del proyecto.

\paragraph*{Capítulo 4. Metodología.}
Se detallarán las prácticas y metodologías de desarrollo empleadas para la realización del videojuego y la razón de su uso.

\paragraph*{Capítulo 5. Planificación y Seguimento.}
Detallaremos la planificación y seguimiento usados en cada una de las etapas del proyecto. 

\paragraph*{Capítulo 6. Análisis de requisitos.}
Se comentará el análisis de requisitos para este proyecto y se explicará en detalle cada uno de los mismos.

\paragraph*{Capítulo 7. Diseño e implementación.}
En este capítulo explicaremos los detalles del diseño y de la implementación de ciertas partes del programa que consideramos más relevantes.

\paragraph*{Capítulo 8. Recepción, conclusión y trabajo futuro.}
En este apartado mostraremos los comentarios obtenidos por la comunidad durante y tras el desarrollo del juego, se relatarán las conclusiones obtenidas y se detallarán los posibles cambios, mejoras y extras que se podrán tener en cuenta en el futuro para su mejora.
