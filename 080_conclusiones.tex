\chapter{Conclusiones y trabajo futuro}

En este último capítulo detallaremos las conclusiones sacadas de la elaboración del proyecto y el posible desarrollo futuro del mismo.

\section{Conclusiones}

Realizar un proyecto de estas características no es sencillo. Dejando a un lado la parte técnica y de diseño, que nunca es trivial, en este caso se requiere un gran grado de investigación, análisis, asimilación del \textit{feedback} y determinación para poder sacarlo adelante. Sin embargo, también es cierto que durante su desarrollo siempre me he encontrado a una comunidad muy ilusionada por la existencia del proyecto y encantada de colaborar y resolver mis dudas.
Es una pena que algo tan esencial como son la mayoría de elementos tecnológicos de hoy en día (y su \textit{software} en particular) todavía no estén adaptados para que toda la parte de la población pueda usarlos sin grandes restricciones. Espero que con este proyecto haya gente que se anime a intentar hacer algo similar (o mejorar lo ya creado) y que entre todos podamos formar una sociedad donde no haya tanta discriminación tecnológica.

\section{Trabajo Futuro}

Cuando se crea cualquier tipo de \textit{software}, siempre existen mejoras y nuevas funcionalidades que puede ser añadidas para mejorar lo creado, sobre todo si se trata de un juego de \textit{software libre}, dado que las comunidades de desarrolladores y jugadores suelen ser bastante apasionadas. Ésta es la razón por la que, desde un principio, hemos establecido los elementos básicos a desarrollar para luego incluir otros elementos que no son tan esenciales.

Durante el desarrollo del proyecto se han tenido muchas ideas que hemos ido recopilando e implementando en caso de que las considerásemos necesarias. Actualmente todas estas ideas están recogidas en la página del proyecto en Github,\footnote{\url{https://github.com/dpenas/roomsgame/issues}} donde cualquier persona puede añadir su comentario o incluso realizar el cambio necesario y crear una \textit{pull request} para que sea incluido en el juego final.

De entre éstas, alguna de las ideas más interesantes con las que podemos mejorar el juego en el futuro son las siguientes: 

\paragraph{Creación de un menú de configuración:} En la actualidad no disponemos de un menú porque podemos realizar todo lo necesario dentro del propio juego. Sin embargo, y a medida que la cantidad de opciones de configuración crece, sería conveniente su creación para que nos mostrase lo que podemos cambiar sin tener que entrar en el juego en sí.

\paragraph{Añadir más variedad de enemigos y elementos:} La cantidad de enemigos diferentes que tenemos actualmente no es muy grande ya que desde el punto de vista académico aumentar dicha variedad no nos aportaba nada. Sin embargo, la experiencia de juego se beneficiaría teniendo más enemigos y objetos con los que interactuar, aunque sin olvidar que deben escalar de manera apropiada.

\paragraph{Integrar un modo historia:} El juego actual es abierto en el sentido de que las partidas son independientes entre sí y no hay un principio ni final definidos, sino que el usuario es el que crea, en cierta manera, su historia. 
No estaría de más integrar un nuevo modo de juego en el que se desarrolle una historia con argumento. Para ello se añadirían ciertos niveles y pantallas predefinidos mediante los cuales se hilvanaría la historia para aquellos usuarios que prefieran este tipo de aventuras.

\paragraph{Crear un tutorial:} Las instrucciones del juego se encuentran disponibles en la página de Github del proyecto, pero no estaría de más la creación de un tutorial o pantalla de inicio que explicase al usuario qué hacer y cómo desenvolverse por los entornos del juego.

\paragraph{Integrar un ``modo resumen'':} Con todas las frases que generamos durante la partida podríamos crear una especie de historia que relatase, en la medida de lo posible, lo que el jugador ha hecho durante dicha partida, dándole cierta importancia a los ``hitos'' conseguidos durante la misma. Otros \textit{roguelikes} como \textit{Dwarf Fortress} disponen de este elemento.\footnote{\url{http://www.bay12games.com/dwarves/}}

\paragraph{Incluir elementos sonoros:} Algunos juegos para invidentes solamente tienen elementos auditivos para informar al usuario sobre lo que tiene a su alrededor. En nuestro caso, al poder crear frases en base a unas gramáticas dadas, no resulta ser algo necesario, pero sí que podríamos crear una opción para que (en vez de) usar siempre descripciones, podamos reproducir sonidos a diferentes grados de volumen e incluso empleando técnicas de sonido tridimensional para informar al jugador sobre distintos elementos y enriquecer la atmósfera del juego. Por ejemplo, acompañar las descripciones de los combates con entrechocar de espadas.

\paragraph{Añadir más complejidad a las gramáticas:} Los textos generados se basan en las gramáticas que se hayan definido para el sistema, pero hay estructuras del lenguaje (como frases reflexivas), que son más complejas de crear que otras. Podríamos introducir estos nuevos elementos para que la cantidad de frases generadas y su variedad sea todavía mayor. 

\paragraph{Integrar recursos lógicos de terceros:} Tener un diccionario propio del sistema para el idioma en el que vayamos a jugar es necesario, dado que no todos los idiomas disponen del mismo tipo de recursos lingüísticos y su uso limitaría la cantidad de idiomas con los que podríamos jugar. Sin embargo, si el juego está, por ejemplo, en inglés, sí que podríamos usar recrsos como la base de datos léxica \textit{WordNet} para que, en vez de usar la palabra que se encuentra en nuestro diccionario, busque sinónimos en dichas bibliotecas, lo que aumentaría de manera significante la riqueza y expresividad del sistema. 

\paragraph{Añadir opción de jugar en base a una semilla:} Algunos \textit{roguelikes} o \textit{roguelites} como \textit{The Binding of Isaac: Rebirth},\footnote{\url{http://store.steampowered.com/app/250900}} permiten que el jugador introduzca un código que equivale a una semilla de aleatoriedad para que los elementos generados sean en base a esa semilla. Esto permitiría que los usuarios compartan partidas donde todo lo generado sea igual para ambos.