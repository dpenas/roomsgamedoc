\chapter{Conclusiones y trabajo futuro}

En este último capítulo detallaremos las conclusiones sacadas de la elaboración de \textit{The accessible dungeon} y el posible desarrollo futuro del mismo. Es un juego en software libre con licencia GNU General Public Licence v3\footnote{\href{https://github.com/dpenas/roomsgame}{Se encuentra disponible en Github}}.

\section{Conclusiones}

Realizar un proyecto de estas características no es sencillo. El desarrollo de un videojuego puede acabar siendo infinito, dado que siempre hay nuevas características adicionales que se pueden incluir y todos los usuarios tienen bastante \textit{feedback} sobre el siguiente paso a tomar. Sin embargo, y gracias a tener claro los puntos que quisimos cubrir en un principio, hemos sido capaces centrarnos en lo esencial para sacarlo adelante y, al mismo tiempo, hemos añadido elementos que los jugadores invidentes consideraban importantes.
Dejando a un lado la parte técnica y de diseño, que nunca es trivial y de la que he aprendido mucho para crearlo de la forma más genérica y extensible posible, en este proyecto se ha requierdo un alto grado de investigación y análisis previos, asimilación del \textit{feedback} así como determinación para poder sacarlo adelante. Sin embargo, también es cierto que durante su desarrollo siempre me he encontrado a una comunidad muy ilusionada por la existencia del proyecto y encantada de colaborar y resolver mis dudas.
Es una pena que algo tan esencial como son la mayoría de elementos tecnológicos de hoy en día (y su \textit{software} en particular) todavía no estén adaptados para que todos los sectores posibles de la población pueda usarlos sin grandes restricciones. Espero que con este proyecto haya gente que se anime a intentar hacer algo similar (o mejorar lo ya creado) y que entre todos podamos formar una sociedad donde no haya tanta discriminación tecnológica.

\section{Trabajo Futuro}

Cuando se crea cualquier tipo de \textit{software}, siempre existen mejoras y nuevas funcionalidades que puede ser añadidas para mejorar lo creado, sobre todo si se trata de un juego de \textit{software libre}, dado que las comunidades de desarrolladores y jugadores suelen ser bastante apasionadas. Ésta es la razón por la que, desde un principio, hemos establecido los elementos básicos a desarrollar para luego incluir otros elementos que no son tan esenciales.

Durante el desarrollo del proyecto se han tenido muchas ideas que hemos ido recopilando e implementando en caso de que las considerásemos necesarias. Actualmente todas estas ideas están recogidas en la página del proyecto en Github,\footnote{\url{https://github.com/dpenas/roomsgame/issues}} donde cualquier persona puede añadir su comentario o incluso realizar el cambio necesario y crear una \textit{pull request} para que sea incluido en el juego final.

De entre estas ideas, algunas de las más interesantes con las que se podría mejorar el juego en el futuro son las siguientes: 

\paragraph{Creación de un menú externo de configuración:} En la actualidad no disponemos de un menú porque podemos llevar a cabo todas las operaciones necesarias dentro del propio juego. Sin embargo, y a medida que la cantidad de opciones de configuración crece, sería conveniente contar con un menú de ese tipo para que nos mostrase lo que podemos cambiar sin tener que entrar en el juego en sí.

\paragraph{Añadir más variedad de enemigos y otros elementos del juego:} La cantidad de tipos de enemigos diferentes que tenemos actualmente no es muy grande ya que desde el punto de vista académico aumentar dicha variedad no nos aportaba nada. Sin embargo, la experiencia de juego se beneficiaría teniendo más enemigos así como objetos con los que interactuar (armas, pociones, etc.), aunque sin olvidar que deben escalar de manera apropiada.

\paragraph{Integrar un ``modo historia'':} El juego actual es \textit{abierto} en el sentido de que las mazmorras son independientes entre sí y no hay un principio ni final definidos, sino que el usuario es el que crea, en cierta manera, su historia. 
Para aquellos usuarios que prefieran este tipo de aventuras, no estaría de más integrar un nuevo modo de juego en el que se desarrolle una historia con un argumento. Para ello se añadirían ciertos niveles y pantallas predefinidos mediante los cuales se hilvanaría la historia.\cite{Brathwaite_Schreiber_2009_ch13}

\paragraph{Crear un tutorial:} Las instrucciones del juego se encuentran disponibles en la página de Github del proyecto, pero no estaría de más la creación de un tutorial o pantalla de inicio que explicase al usuario qué hacer y cómo desenvolverse dentro del juego.

\paragraph{Integrar un ``modo resumen'':} A partir de todos los sucesos acaecidos a lo largo de la partida y sus correspondientes descripciones generadas, podríamos crear una especie de historia que relatase, en la medida de lo posible, lo que el jugador ha hecho durante dicha partida, dándole cierta importancia a los ``hitos'' conseguidos durante la misma. Otros \textit{roguelikes} como \textit{Dwarf Fortress} disponen de este elemento\footnote{\url{http://www.bay12games.com/dwarves/}}. Creemos que éste sería uno de los puntos más complejos, pues estaríamos entrando dentro del \emph{storytelling}\cite{Salen_and_Zimmerman_2004a_ch26} teniendo que recurrir a técnicas más complejas de generación automática del lenguaje\cite{Reiter_and_Dale_2000b} e incluso modelización de técnicas narrativas \cite{Mani2012a}, lo que supondría todo un nuevo proyecto en sí.

\paragraph{Incluir elementos sonoros:} Algunos juegos para invidentes solamente tienen elementos auditivos para informar al jugador sobre lo que tiene a su alrededor dentro del juego. En nuestro caso, al poder generar descripciones detalladas, no resulta necesario, pero sí que podríamos crear una opción para que (en vez de) usar siempre descripciones, podamos reproducir sonidos a diferentes grados de volumen e incluso empleando técnicas de sonido tridimensional para informar al jugador sobre distintos elementos del juego, así como enriquecer la atmósfera del mismo. Un ejemplo sencillo sería el de acompañar las descripciones de los combates con entrechocar de espadas o que se pueda oír los leves gruñidos de un enemigo que parece estar escondido hacia el fondo de la estancia.

\paragraph{Añadir más complejidad a las gramáticas:} Los textos generados se basan en las gramáticas que se hayan definido para el sistema, pero hay estructuras del lenguaje (como frases reflexivas, oraciones subordinadas, etc.), que son más complejas de crear que otras. Podríamos introducir algunos de estos elementos para que la capacidad y variedad expresivas del sistema sea todavía mayor. 

\paragraph{Integrar recursos lingüísticos de terceros:} Es necesario tener un diccionario propio del sistema para el idioma en el que vayamos a jugar, dado que no todos los idiomas disponen del mismo tipo de recursos lingüísticos y su uso limitaría la cantidad de idiomas con los que podríamos jugar. Sin embargo, si el juego está, por ejemplo, en inglés, sí que podríamos usar recrsos como la base de datos léxica \textit{WordNet}\cite{MilBecFelGroMil90a} para que, en vez de usar la palabra que se encuentra en nuestro diccionario, busque sinónimos en dichas bibliotecas, lo que aumentaría de manera significante la riqueza y expresividad del sistema. 

\paragraph{Añadir la opción de jugar en base a una semilla:} Algunos \textit{roguelikes} y \textit{roguelites} como \textit{The Binding of Isaac: Rebirth},\footnote{\url{http://store.steampowered.com/app/250900}} permiten que el jugador introduzca un código que equivale a una semilla de aleatoriedad y que los elementos generados lo sean en base a esa semilla.\cite{Betts2014a} Esto permitiría que los usuarios compartan partidas donde todo lo generado sea igual para ambos.