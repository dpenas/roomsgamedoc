\chapter{Recepción, conclusión y trabajo futuro}

En este último capítulo detallaremos la recepción y el \textit{feedback} recibido tras mostrar nuestro producto a la comunidad, la conclusión sacadas de la elaboración del proyecto y el posible trabajo futuro del mismo.

\section{Recepción y \textit{Feedback}}

TODO:

\section{Conclusión}

Realizar un proyecto de estas características no es sencillo. Dejando a un lado la parte técnica y de diseño, que nunca es trivial, en este caso se requiere un gran grado de investigación, análisis, asimilación del \textit{feedback} y determinación para poder sacarlo adelante, pero durante su desarrollo siempre me he encontrado a una comunidad muy ilusionada por su existencia y encantada de colaborar y resolver mis dudas. 
Es una pena que algo tan esencial como son la mayoría de elementos tecnológicos de hoy en día (y su \textit{software} en particular) todavía no estén adaptados para que toda la parte de la población pueda usarlos y sean dejados de lado, muchas veces por mera ignorancia. Espero que con este proyecto haya gente que se anime a intentar hacer algo similar (o mejorar lo ya creado) y que entre todos podamos formar una sociedad donde no haya tanta discriminación tecnológica.

\section{Trabajo Futuro}

Cuando se crea cualquier tipo de \textit{software}, siempre existen mejoras y nuevas funcionalidades que puede ser añadidas para mejorar lo creado, sobre todo si se trata de un juego de \textit{software libre}, dado que las comunidades de desarrolladores y jugadores suelen ser bastante apasionadas. Esta es la razón por la que, desde un principio, hemos definido lo que queremos hacer como básico para luego incluir otros elementos que no son tan esenciales.

Durante el desarrollo del proyecto se han tenido muchas ideas que hemos ido recopilando e implementando en caso de que las considerásemos necesarias. Actualmente todas estas ideas se encuentran en la página del proyecto en Github\footnote{\url{https://github.com/dpenas/roomsgame/issues}}, donde cualquier persona puede añadir su comentario o incluso realizar el cambio necesario y crear una \textit{pull request} para que sea incluido en el juego final.

Algunas de estas ideas con las que podemos mejorar el juego en el futuro son las siguientes: 

\paragraph{Creación de un menú:} En la actualidad no tenemos un menú porque podemos realizar todo lo necesario dentro del propio juego. Sin embargo, y a medida que la cantidad de opciones crece, sería conveniente su creación para que nos mostrase lo que podemos cambiar sin tener que entrar en el juego en sí.

\paragraph{Añadir más variedad de enemigos y elementos:} La cantidad de enemigos que tenemos no es muy grande y el juego se beneficiaría teniendo más enemigos y objetos con los que interactuar, sin olvidar que deben de escalar de manera apropiada.

\paragraph{Crear un modo historia:} El juego actual es libre. Esto quiere decir que las partidas son independientes entre sí y no hay un principio y final definido, sino que el usuario es el que crea, en cierta manera, su historia. 
No estaría de más tener un modo de juego diferente que lleve al usuario a ciertos niveles y pantallas predefinidas para aquellos usuarios que prefieran este tipo de aventuras.

\paragraph{Crear un tutorial:} La información sobre cómo empezar a jugar se encuentra disponible en la página de Github del proyecto, pero no estaría de más la creación de un tutorial o pantalla de inicio que explicase al usuario qué hacer y cómo desenvolverse por los entornos del juego.

\paragraph{Crear un ``modo resumen'':} Con todas las frases que generamos durante la partida podríamos crear una especie de historia que relatase, en la medida de lo posible, lo que el jugador ha hecho durante dicha partida, dándole cierta importancia a los ``hitos'' conseguidos durante la misma.

\paragraph{Incluir elementos sonoros:} Algunos juegos para invidentes solamente tienen elementos auditivos para informar al usuario sobre lo que tiene a su alrededor. En nuestro caso, al poder crear frases en base a unas gramáticas dadas, no resulta ser algo necesario, pero sí que podríamos crear una opción para que, en vez de usar siempre descripciones, podamos reproducir sonidos a diferentes grados de volumen para informar al jugador sobre distintos elementos.

\paragraph{Añadir más complejidad a las gramáticas:} Las frases creadas se basan en gramáticas dadas, pero hay algunas particularidades (como frases reflexivas), que son más complejas de crear. Podríamos introducir esta opción para que la cantidad de frases generadas y su variedad sea todavía mayor. 

\paragraph{Añadir más complejidad a los diccionarios:} Tener un diccionario propio es necesario, dado que no todos los idiomas disponen del mismo tipo de bibliotecas y su uso limitaría cantidad de idiomas con los que podremos jugar. Sin embargo, si el juego está en inglés o español, sí que podríamos usar alguna biblioteca para que, en vez de usar la palabra que se encuentra en nuestro diccionario, busque un sinónimos en dichas bibliotecas, lo que aumentaría de manera significante la variedad de las palabras que usamos. 

\paragraph{Crear opción de jugar en base a una semilla:} Algunos \textit{roguelikes} o \textit{roguelites} como \textit{The Binding of Isaac: Rebirth} \footnote{\url{http://store.steampowered.com/app/250900}}, permiten que el jugador introduzca un código que equivale a una semilla de aleatoriedad para que los elementos generados sean en base a esa semilla. Esto permitiría que los usuarios compartan partidas donde todo lo generado sea igual para ambos.