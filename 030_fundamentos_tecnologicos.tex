\chapter{Fundamentos Tecnológicos}

En este capítulo hablaremos sobre los fundamentos tecnológicos que vamos a usar en este proyecto y, si cabe, la razón por la que fueron elegidas. En primer lugar citaremos las herramientas que hemos usado y, en segundo lugar, las bibliotecas que hemos decidido utilizar en el programa en sí.

\section{Herramientas empleadas}

\paragraph{Java} Lenguaje de programación orientado a objetos cuya primera aparición fue en 1995. Es uno de los lenguajes de programación más utilizados en la industria y una de sus principales características es que es multiplataforma, es decir, puede ser ejecutado en cualquier sistema operativo que tenga la \textit{Java Virtual Machine} instalada sin necesidad de realizar cambios en el código (WORA\footnote{\textit{Write once, run anywhere}. Eslogan creado por Sun Microsystems para mostrar los beneficios de la multiplataforma}). Esta ventaja es esencial en nuestro caso, dado que la mayoría de las personas que pueden estar interesadas en el proyecto usan una gran variedad de sistemas operativos.

 \paragraph{Eclipse} Es un IDE\footnote{\textit{Integrated Development Environment}. Entorno de desarrollo integrado} usado para escribir código en múltiples idiomas. También incluye una serie de \textit{plugins} que facilitan y automatizan muchas de las labores a realizar como el uso de sistema de controles, ejecución de código y tests, herramientas de debug, autocompletado de código, etc.

 \paragraph{Git} Sistema de control de versiones distribuido introducido en 2005 y desarrollado principalmente por Linus Torvalds. Es el control de versiones referencia en la mayoría de empresas y proyectos de software libre gracias a su rapidez y, al ser distribuida, permite trabajar y realizar \textit{commits} del código sin necesidad de conexión a internet.

 \paragraph{GitHub} Plataforma de desarrollo colaborativo usada para alojar proyectos usando el sistema de control de versiones Git. La mayoría de proyectos de código abierto lo usan, dado que es gratuito, aunque también tiene la opción de almacenar el código de forma privada tras, previamente, realizar un pago.

 \paragraph{Listas de Correo} Las listas de correo son un método de comunicación muy usado por diferentes comunidades, especialmente en el desarrollo de software, que ayudan a los usuarios que participan en ellas a enviar correos a múltiples personas que lo deseen de forma anónima y, al mismo tiempo, tener un historial de las respuestas dadas por los mismos. En nuestro caso la hemos usado para comunicarnos con un grupo de usuarios y desarrolladores de videojuegos para invidentes.

 \paragraph{Reddit} Web creada en 2005 y que actualmente se encuentra en el top 50 de las más visitadas del mundo. Cuenta con una comunidad gigante que está dividida en muchísimos subgrupos dependiendo del tema a tratar. La hemos usado como una herramienta de feedback. Especialmente los \textit{subreddits} de daltónicos\url{https://www.reddit.com/r/ColorBlind/} y gente ciega\url{https://www.reddit.com/r/blind/}. 

\paragraph{Dia} Aplicación informática que permite la creación de todo tipo de diagramas. En nuestro caso lo hemos usado para crear los diagramas UML que se encuentran en esta memoria.

\paragraph{Gantt Project} Programa de software libre diseñado para la creación de diagramas de Gantt.

\paragraph{JSON} JavaScript Object Notation. Es un formato muy usado en APIs para intercambio de datos, similar a XML. En nuestro caso lo usamos para definir las gramáticas y diccionarios de nuestro proyecto, dado que es muy sencillo de leer y especificar. Hay numerosas bibliotecas que nos permiten analizar y trabajar con este formato en Java. La que nosotros usamos es \textit{Gson}.

\paragraph{\LaTeX} Sistema de composición de textos altamente usado por la mayoría de textos científicos dada la facilidad de su composición, simpleza, alta calidad y herramientas que a ayudan a la creación de fórmulas, inserción de imágenes y muchos otros elementos. Muy modificable. Es el sistema que hemos usado para la creación de este documento.

\paragraph{NVDA} Lector de pantalla de código libre para Windows. Orca es, en cierta medida, su equivalente en Linux.

\section{Bibliotecas empleadas}

\paragraph{Gson} Biblioteca usada para transformar archivos JSON a objetos de Java y viceversa.

\paragraph{JCurses} JCurses\footnote{\textit{The Java Curses Library}} es una biblioteca para el desarrollo de aplicaciones de terminal para JAVA. Es similar a AWT\footnote{\textit{Abstract Window Toolkit.} Kit de herramientas de interfaz de usuario de la plataforma original de Java}, pero basada en el sistema de ventanas Curses de UNIX.

\paragraph{Libjcsi} Biblioteca de representación gráfica que trabaja sobre JCurses y simplifica la tarea de representar y refrescar elementos del terminal.
